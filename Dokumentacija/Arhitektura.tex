\chapter{Arhitektura i dizajn sustava}
		
		%dio 1. revizije

		%Potrebno je opisati stil arhitekture te identificirati: podsustave, preslikavanje na radnu platformu, spremišta podataka, mrežne protokole, globalni upravljački tok i sklopovsko-programske zahtjeve. Po točkama razraditi i popratiti odgovarajućim skicama:
	% - izbor arhitekture temeljem principa oblikovanja pokazanih na           predavanjima (objasniti zašto ste baš odabrali takvu arhitekturu)
	% - organizaciju sustava s najviše razine apstrakcije (npr. klijent-       poslužitelj, baza podataka, datotečni sustav, grafičko sučelje)
	% - organizaciju aplikacije (npr. slojevi frontend i backend, MVC          arhitektura)		
	

					
		\section{Baza podataka}
			
			%dio 1. revizije
			
		%Potrebno je opisati koju vrstu i implementaciju baze podataka ste odabrali, glavne komponente od kojih se sastoji i slično.
		
			\subsection{Opis tablica}
			

				%Svaku tablicu je potrebno opisati po zadanom predlošku. Lijevo se nalazi točno ime varijable u bazi podataka, u sredini se nalazi tip podataka, a desno se nalazi opis varijable. Svjetlozelenom bojom označite primarni ključ. Svjetlo plavom označite strani ključ
				
				
				\begin{longtblr}[
					label=none,
					entry=none
					]{
						width = \textwidth,
						colspec={|X[6,l]|X[6, l]|X[20, l]|}, 
						rowhead = 1,
					} %definicija širine tablice, širine stupaca, poravnanje i broja redaka naslova tablice
					\hline \SetCell[c=3]{c}{\textbf{korisnik - ime tablice}}	 \\ \hline[3pt]
					\SetCell{LightGreen}IDKorisnik & INT	&  	Lorem ipsum dolor sit amet, consectetur adipiscing elit, sed do eiusmod  	\\ \hline
					korisnickoIme	& VARCHAR &   	\\ \hline 
					email & VARCHAR &   \\ \hline 
					ime & VARCHAR	&  		\\ \hline 
					\SetCell{LightBlue} primjer	& VARCHAR &   	\\ \hline 
				\end{longtblr}
				
				
			
			\subsection{Dijagram baze podataka}
				% U ovom potpoglavlju potrebno je umetnuti dijagram baze podataka. Primarni i strani ključevi moraju biti označeni, a tablice povezane. Bazu podataka je potrebno normalizirati.
			
			\eject
			
			
		\section{Dijagram razreda}
		
			%Potrebno je priložiti dijagram razreda s pripadajućim opisom. Zbog preglednosti je moguće dijagram razlomiti na više njih, ali moraju biti grupirani prema sličnim razinama apstrakcije i srodnim funkcionalnostima.
			
			%dio 1. revizije
			
			%Prilikom prve predaje projekta, potrebno je priložiti potpuno razrađen dijagram razreda vezan uz generičku funkcionalnost sustava. Ostale funkcionalnosti trebaju biti idejno razrađene u dijagramu sa sljedećim komponentama: nazivi razreda, nazivi metoda i vrste pristupa metodama (npr. javni, zaštićeni), nazivi atributa razreda, veze i odnosi između razreda.
			
			%dio 2. revizije			
			
			%Prilikom druge predaje projekta dijagram razreda i opisi moraju odgovarati stvarnom stanju implementacije
			
			
			
			\eject
		
		\section{Dijagram stanja}
			
			
			%dio 2. revizije
			
			%Potrebno je priložiti dijagram stanja i opisati ga. Dovoljan je jedan dijagram stanja koji prikazuje značajan dio funkcionalnosti sustava. Na primjer, stanja korisničkog sučelja i tijek korištenja neke ključne funkcionalnosti jesu značajan dio sustava, a registracija i prijava nisu.
			
			
			\eject 
		
		\section{Dijagram aktivnosti}
			
			%dio 2. revizije
			
			 %Potrebno je priložiti dijagram aktivnosti s pripadajućim opisom. Dijagram aktivnosti treba prikazivati značajan dio sustava.
			
			\eject
		\section{Dijagram komponenti}
		
			%dio 2. revizije
		
			 %Potrebno je priložiti dijagram komponenti s pripadajućim opisom. Dijagram komponenti treba prikazivati strukturu cijele aplikacije.