\chapter{Arhitektura i dizajn sustava}
		
		%dio 1. revizije

		%Potrebno je opisati stil arhitekture te identificirati: podsustave, preslikavanje na radnu platformu, spremišta podataka, mrežne protokole, globalni upravljački tok i sklopovsko-programske zahtjeve. Po točkama razraditi i popratiti odgovarajućim skicama:
	% - izbor arhitekture temeljem principa oblikovanja pokazanih na           predavanjima (objasniti zašto ste baš odabrali takvu arhitekturu)
	% - organizaciju sustava s najviše razine apstrakcije (npr. klijent-       poslužitelj, baza podataka, datotečni sustav, grafičko sučelje)
	% - organizaciju aplikacije (npr. slojevi frontend i backend, MVC          arhitektura)		
	

					
		\section{Baza podataka}
			
			%dio 1. revizije
			
		%Potrebno je opisati koju vrstu i implementaciju baze podataka ste odabrali, glavne komponente od kojih se sastoji i slično.
        Baza podataka koja se koristi u projektu "Ozdravi" ključna je komponenta sustava koja omogućava pohranu, upravljanje i praćenje svih relevantnih informacija o korisnicima i njihovim zdravstvenim podacima. Odabrana je relacijska baza podataka koja se sastoji od nekoliko ključnih entiteta:
            \begin{packed_item}
                \item Person
                \item Examination
                \item SecondOpinion
                \item Instruction
                \item SickLeaveRecommendation
                \item LoginCreds
            \end{packed_item}

        Baza podataka "Ozdravi" omogućava integraciju svih informacija kako bi podržala procese komunikacije, pregleda i dijagnoza u kontekstu brige o zdravstvenim potrebama djece. Ova struktura omogućava učinkovito praćenje i upravljanje svim relevantnim medicinskim podacima i podržava glavne funkcionalnosti sustava.
        
			\subsection{Opis tablica}

            \textbf{Osoba} Ovaj entitet sadržava osnovne informacije o svim korisnicima sustava, uključujući njihovu ulogu, identifikacijske brojeve, ime, prezime, korisničko ime i lozinku. Ova tablica je ključna za autentikaciju korisnika i definira njihove uloge unutar sustava.
				\begin{longtblr}[
					label=none,
					entry=none
					]{
						width = \textwidth,
						colspec={|X[6,l]|X[6, l]|X[20, l]|}, 
						rowhead = 1,
					} %definicija širine tablice, širine stupaca, poravnanje i broja redaka naslova tablice
					\hline \SetCell[c=3]{c}{\textbf{Person (osoba)}}	 \\ \hline[3pt]
					\SetCell{LightGreen}id & INT	&  	- 	\\ \hline
					\SetCell{LightBlue}parent\_id	& INT & ID Osobe koja je roditelj osobe [Parent] \\ \hline 
                    \SetCell{LightBlue}doctor\_id	& INT & ID Osobe koja je doktor osobe [Doctor/Pediatrician] \\ \hline 
                    \SetCell{LightBlue}address\_id	& INT & ID Adrese osobe \\ \hline 
					OIB & VARCHAR & Personal ID Number \\ \hline 
					first\_name & VARCHAR &  Ime osobe (NOT NULL) \\ \hline 
                    last\_name & VARCHAR &  Prezime osobe (NOT NULL) \\ \hline 
                    role & VARCHAR &  Uloga korisnika (NOT NULL) \\ \hline 
                    password & VARCHAR &  Lozinka za prijavu u sustav \\ \hline
                    email & VARCHAR &  e-mail za prijavu u sustav \\ \hline 
					
				\end{longtblr}

                \textbf{Pregled} Ovaj entitet sadržava podatke o pacijentima, uključujući informacije o njihovim pregledima, dijagnozama i zapisnicima. Sadrži veze između pacijenata, doktora i pregleda, omogućavajući praćenje medicinskih podataka.
                
                \begin{longtblr}[
					label=none,
					entry=none
					]{
						width = \textwidth,
						colspec={|X[6,l]|X[6, l]|X[20, l]|}, 
						rowhead = 1,
					} 
					\hline \SetCell[c=3]{c}{\textbf{Examination (pregled)}}	 \\ \hline[3pt]
					\SetCell{LightGreen}id & INT	&  	- 	\\ \hline
					\SetCell{LightBlue}patient\_id	& INT & ID Osobe koja je pacijent [Parent/Child] (NOT NULL) \\ \hline 
                    \SetCell{LightBlue}doctor\_id	& INT & ID Osobe koja vrši pregled [Doctor/Pediatrician] (NOT NULL) \\ \hline 
                    \SetCell{LightBlue}scheduler\_id	& INT & ID osobe koja je zakazala pregled [Doctor/Pediatriciran] (NOT NULL) \\ \hline 
                    \SetCell{LightBlue}address\_id	& INT & ID adrese na kojoj se održava pregled \\ \hline 
					report & TEXT & Dijagnoza/zapisnik s pregleda/... \\ \hline 
					date & DATETIME &  Datum i vrijeme pregleda (NOT NULL) \\ \hline 
				\end{longtblr}

                \textbf{Drugo mišljenje} Ovaj entitet služi za praćenje detalja o pregledima pacijenata. Sadrži informacije o osobama koje su zatražile drugo mišljenje, doktorima koji su dali mišljenje te samom sadržaju mišljenja. To omogućava praćenje medicinskih konzultacija.
                
                 \begin{longtblr}[
					label=none,
					entry=none
					]{
						width = \textwidth,
						colspec={|X[6,l]|X[6, l]|X[20, l]|}, 
						rowhead = 1,
					} 
					\hline \SetCell[c=3]{c}{\textbf{SecondOpinion (DrugoMišljenje)}}	 \\ \hline[3pt]
					\SetCell{LightGreen}id & INT	&  	- 	\\ \hline
					\SetCell{LightBlue}requester\_id	& INT & ID Osobe koja je zatražila drugo mišljenje [Parent] (NOT NULL) \\ \hline 
                    \SetCell{LightBlue}doctor\_id & INT & ID Osobe koja je dala drugo mišljenje [Doctor/Pediatrician] (NOT NULL) \\ \hline 
					second_opinion & TEXT & Drugo mišljenje doktora (NOT NULL)\\ \hline 
					content & TEXT &  Sadržaj na koji se daje drugo mišljenje (nalaz iz privatne ustanove) \\ \hline 
				\end{longtblr}

                \textbf{Uputa} Ovaj entitet sadržava informacije o izdanim uputama, uključujući informacije o pacijentima, doktorima koji stvaraju upute i statusu upute. Omogućava praćenje i upravljanje medicinskim preporukama.
                
                \begin{longtblr}[
					label=none,
					entry=none
					]{
						width = \textwidth,
						colspec={|X[6,l]|X[6, l]|X[20, l]|}, 
						rowhead = 1,
					} 
					\hline \SetCell[c=3]{c}{\textbf{Instruction (uputa)}}	 \\ \hline[3pt]
					\SetCell{LightGreen}id & INT	&  	- 	\\ \hline
					\SetCell{LightBlue}doctor\_id	& INT & ID Osobe koja je izdala uputu [Doctor/Pediatrician] (NOT NULL) \\ \hline 
                    \SetCell{LightBlue}patient\_id & INT & ID Osobe za koju je uputa [Parent] (NOT NULL) \\ \hline 
					date & DATETIME & Datum i vrijeme izdavanja upute (NOT NULL)\\ \hline 
					content & TEXT &  Sadržaj uput (NOT NULL) \\ \hline 
				\end{longtblr}

                \textbf{Preopurka za bolovanje} Ovaj enitet sadrži informacije o izdanim preporukama za bolovanje.  
                \begin{longtblr}[
					label=none,
					entry=none
					]{
						width = \textwidth,
						colspec={|X[6,l]|X[6, l]|X[20, l]|}, 
						rowhead = 1,
					} 
					\hline \SetCell[c=3]{c}{\textbf{SickLeaveRecommendation (Preporuka za bolovanje)}}	 \\ \hline[3pt]
					\SetCell{LightGreen}id & INT	&  	- 	\\ \hline
                    \SetCell{LightBlue}patient\_id & INT & ID Osobe za koju se traži bolovanje [Parent] (NOT NULL) \\ \hline 
                    \SetCell{LightBlue}creator\_id & INT & ID Osobe koja je stvorila preporuku [Pediatrician] (NOT NULL) \\ \hline
                    \SetCell{LightBlue}approver\_id & INT & ID Osobe koja mora odobriti preporuku [Doctor] (NOT NULL) \\ \hline 
					\SetCell{LightBlue}examination\_id	& INT & ID pregleda na temelju kojeg se izdaje preporuka (NOT NULL) \\ \hline 
					email & VARCHAR & e-mail na koji se šalje, ako je potvrđena (NOT NULL) \\ \hline 
					status & BOOLEAN &  Je li preporuka odobrena \\ \hline 
				\end{longtblr}

			\subsection{Dijagram baze podataka}
				% U ovom potpoglavlju potrebno je umetnuti dijagram baze podataka. Primarni i strani ključevi moraju biti označeni, a tablice povezane. Bazu podataka je potrebno normalizirati.
			
			\eject
			
			
		\section{Dijagram razreda}
		
			%Potrebno je priložiti dijagram razreda s pripadajućim opisom. Zbog preglednosti je moguće dijagram razlomiti na više njih, ali moraju biti grupirani prema sličnim razinama apstrakcije i srodnim funkcionalnostima.
			
			%dio 1. revizije
			
			%Prilikom prve predaje projekta, potrebno je priložiti potpuno razrađen dijagram razreda vezan uz generičku funkcionalnost sustava. Ostale funkcionalnosti trebaju biti idejno razrađene u dijagramu sa sljedećim komponentama: nazivi razreda, nazivi metoda i vrste pristupa metodama (npr. javni, zaštićeni), nazivi atributa razreda, veze i odnosi između razreda.
			
			%dio 2. revizije			
			
			%Prilikom druge predaje projekta dijagram razreda i opisi moraju odgovarati stvarnom stanju implementacije
			
			
			
			\eject
		
		\section{Dijagram stanja}
			
			
			%dio 2. revizije
			
			%Potrebno je priložiti dijagram stanja i opisati ga. Dovoljan je jedan dijagram stanja koji prikazuje značajan dio funkcionalnosti sustava. Na primjer, stanja korisničkog sučelja i tijek korištenja neke ključne funkcionalnosti jesu značajan dio sustava, a registracija i prijava nisu.
			
			
			\eject 
		
		\section{Dijagram aktivnosti}
			
			%dio 2. revizije
			
			 %Potrebno je priložiti dijagram aktivnosti s pripadajućim opisom. Dijagram aktivnosti treba prikazivati značajan dio sustava.
			
			\eject
		\section{Dijagram komponenti}
		
			%dio 2. revizije
		
			 %Potrebno je priložiti dijagram komponenti s pripadajućim opisom. Dijagram komponenti treba prikazivati strukturu cijele aplikacije.