%definira klasu dokumenta 
\documentclass[12pt]{report} 

%prostor izmedu naredbi \documentclass i \begin{document} se zove uvod. U njemu se nalaze naredbe koje se odnose na cijeli dokument

%osnovni LaTex ne može riješiti sve probleme, pa se koriste različiti paketi koji olakšavaju izradu željenog dokumenta
\usepackage[croatian]{babel} 
\usepackage{amssymb}
\usepackage{amsmath}
\usepackage{txfonts}
\usepackage{mathdots}
\usepackage{titlesec}
\usepackage{array}
\usepackage{lastpage}
\usepackage{etoolbox}
\usepackage{tabularray}
\usepackage{color, colortbl}
\usepackage{adjustbox}
\usepackage{geometry}
\usepackage[classicReIm]{kpfonts}
\usepackage{hyperref}
\usepackage{fancyhdr}
\usepackage{listings}
\usepackage{float}
\usepackage{setspace}
\usepackage{xcolor}


\definecolor{mygreen}{rgb}{0,0.6,0}
\definecolor{mygray}{rgb}{0.5,0.5,0.5}
\definecolor{mymauve}{rgb}{0.58,0,0.82}

\lstdefinestyle{JavaStyle}{ %
	language=Java,
	backgroundcolor=\color{white},   % choose the background color
	basicstyle=\footnotesize,        % size of fonts used for the code
	breaklines=true,                 % automatic line breaking only at whitespace
	captionpos=b,                    % sets the caption-position to bottom
	commentstyle=\color{mygreen},    % comment style
	escapeinside={\%*}{*)},          % if you want to add LaTeX within your code
	keywordstyle=\color{blue},       % keyword style
	stringstyle=\color{mymauve},     % string literal style
	frame=single,
	numbers=left,
	numberstyle=\small
}

\definecolor{dkgreen}{rgb}{0,0.6,0}
\definecolor{gray}{rgb}{0.5,0.5,0.5}
\definecolor{mauve}{rgb}{0.58,0,0.82}
\lstdefinestyle{SQLStyle}{
	language=SQL,
	basicstyle={\small\ttfamily},
  	belowskip=3mm,
  	breakatwhitespace=true,
  	breaklines=true,
  	classoffset=0,
  	columns=flexible,
  	commentstyle=\color{dkgreen},
  	framexleftmargin=0.25em,
  	% frameshape={}{yy}{}{}, %To remove to vertical lines on left, set `frameshape={}{}{}{}`
	frame=single,
	keywordstyle=\color{blue},
	numbers=none, %If you want line numbers, set `numbers=left`
	numberstyle=\tiny\color{gray},
	showstringspaces=false,
	stringstyle=\color{mauve},
	tabsize=3,
	xleftmargin =1em,
	morekeywords={user, with, password, database, owner},
	captionpos=b
}

\lstdefinestyle{ShellStyle}{
	language=bash,
	backgroundcolor=\color{white},   % choose the background color
	basicstyle=\footnotesize,        % size of fonts used for the code
	breaklines=true,                 % automatic line breaking only at whitespace
	captionpos=b,                    % sets the caption-position to bottom
	frame=single,
	numbers=none
}

\restylefloat{table}
\patchcmd{\chapter}{\thispagestyle{plain}}{\thispagestyle{fancy}}{}{} %redefiniranje stila stranice u paketu fancyhdr

%oblik naslova poglavlja
\titleformat{\chapter}{\normalfont\huge\bfseries}{\thechapter.}{20pt}{\Huge}
\titlespacing{\chapter}{0pt}{0pt}{40pt}


\linespread{1.3} %razmak između redaka

\geometry{a4paper, left=1in, top=1in,}  %oblik stranice

\hypersetup{ colorlinks, citecolor=black, filecolor=black, linkcolor=black,	urlcolor=black }   %izgled poveznice


%prored smanjen između redaka u nabrajanjima i popisima
\newenvironment{packed_enum}{
	\begin{enumerate}
		\setlength{\itemsep}{0pt}
		\setlength{\parskip}{0pt}
		\setlength{\parsep}{0pt}
	}{\end{enumerate}}

\newenvironment{packed_item}{
	\begin{itemize}
		\setlength{\itemsep}{0pt}
		\setlength{\parskip}{0pt}
		\setlength{\parsep}{0pt}
	}{\end{itemize}}


%boja za privatni i udaljeni kljuc u tablicama
\definecolor{LightBlue}{rgb}{0.9,0.9,1}
\definecolor{LightGreen}{rgb}{0.9,1,0.9}

%Promjena teksta za dugačke tablice
\DefTblrTemplate{contfoot-text}{normal}{Nastavljeno na idućoj stranici}
\SetTblrTemplate{contfoot-text}{normal}
\DefTblrTemplate{conthead-text}{normal}{(Nastavljeno)}
\SetTblrTemplate{conthead-text}{normal}
\DefTblrTemplate{middlehead,lasthead}{normal}{Nastavljeno od prethodne stranice}
\SetTblrTemplate{middlehead,lasthead}{normal}

%podesavanje zaglavlja i podnožja

\pagestyle{fancy}
\lhead{Programsko inženjerstvo}
\rhead{Ozdravi} %naziv projektnog zadatka?
\lfoot{Welebyte}
\cfoot{stranica \thepage/\pageref{LastPage}}
\rfoot{\today}
\renewcommand{\headrulewidth}{0.2pt}
\renewcommand{\footrulewidth}{0.2pt}


\begin{document} 
	\begin{titlepage}
		\begin{center}
			\vspace*{\stretch{1.0}} %u kombinaciji s ostalim \vspace naredbama definira razmak između redaka teksta
			\LARGE Programsko inženjerstvo\\
			\large Ak. god. 2023./2024.\\
			
			\vspace*{\stretch{3.0}}
			
			\huge $$Ozdravi$$\\
			\Large Dokumentacija, Rev. \textit{$2.0$}\\
			
			\vspace*{\stretch{12.0}}
			\normalsize
			Grupa: \textit{Welebyte}\\
			Voditelj: \textit{Dorian Matić}\\
			
			
			\vspace*{\stretch{1.0}}
			%Datum predaje: \textit{$<$17$>$. $<$studenoga$>$. $<$2023$>$.}\\
			Datum predaje: \textit{17. studenoga 2023.}\\
	
			\vspace*{\stretch{4.0}}
			
			Nastavnik: \textit{Ivana Lulić}\\
		
		\end{center}

	
	\end{titlepage}

	\tableofcontents

	\chapter{Dnevnik promjena dokumentacije}
		
		\textbf{\textit{Kontinuirano osvježavanje}}\\
				
		
		\begin{longtblr}[
				label=none
			]{
				width = \textwidth, 
				colspec={|X[2]|X[13]|X[3]|X[4]|}, %sirina celije []
				rowhead = 1
			}
			\hline
			\textbf{Rev.}	& \textbf{Opis promjene/dodatka} & \textbf{Autori} & \textbf{Datum}\\[3pt] \hline
			0.1 & Napravljen predložak	& L. Crvelin & 24.10.2023. 		\\[3pt] \hline 
            0.2	& Napisan opis projektnog zadatka & L. Crvelin & 27.10.2023. 	\\[3pt] \hline
            0.3.1	& Napisani funkcionalni zahtjevi & L. Crvelin & 27.10.2023. 	\\[3pt] \hline
            0.3.2	& Izmijenjeni funkcionalni zahtjevi & L. Crvelin & 31.10.2023. 	\\[3pt] \hline
            0.4.1	& Dodani opisi obrazaca uporabe & L. Crvelin & 2.11.2023. 	\\[3pt] \hline 
			0.4.2   & Izmijenjeni opisi obrazaca uporabe & L. Crvelin & 2.11.2023. \\ [3pt] \hline 
            0.5  & Opis baze, opis tablica & L. Crvelin & 2.11.2023. \\ [3pt] \hline 
            0.6 & Arhitektura i dizajn sustava & D. Matić & 6.11.2023. \\[3pt] \hline 
			0.7	& Dijagram baze podataka, dopisan opis tablica & L. Crvelin & 6.11.2023. 	\\[3pt] \hline 
			0.8 & Dopunjeni opisi obrazaca uporabe & L. Crvelin & 7.11.2023. \\[3pt] \hline
			0.9 & Promijenjen dijagram baze podataka & L. Crvelin & 8.11.2023. \\[3pt] \hline
			0.10.1 & Izmjena opisa obrazaca uporabe & L. Crvelin & 8.11.2023. \\[3pt] \hline
			0.10.2 & Izmjena opisa obrazaca uporabe i tablica entiteta & L. Crvelin & 9.11.2023. \\[3pt] \hline
			0.11 & Dodani dijagrami obrazaca uporabe & L. Crvelin & 11.11.2023. \\[3pt] \hline
			0.12.1 & Započeti sekvencijski dijagrami & L. Crvelin & 11.11.2023. \\[3pt] \hline
			0.12.2 & Završeni sekvencijski dijagrami & L. Crvelin & 12.11.2023. \\[3pt] \hline
			0.13 & Opis ostalih zahtjeva & L. Crvelin & 12.11.2023. \\[3pt] \hline
			0.14.1 & Izmjena i dopuna odlomka "Arhitektura i dizajn sustava" & D.Matić & 13.11.2023. \\[3pt] \hline
			0.14.2 & Uređivanje dokumenta & L. Crvelin & 13.11.2023. \\[3pt] \hline
			0.14.3 & Ispunjena tablica aktivnosti & L. Crvelin & 16.11.2023. \\[3pt] \hline
			\textbf{1.0} & Verzija samo s bitnim dijelovima za 1. ciklus & L. Crvelin & 17.11.2023. \\[3pt] \hline 
			1.1 & Izmijenjini opisi obrazaca uporabe i dijagrami obrazaca uporabe & L. Crvelin & 7.12.2023. \\[3pt] \hline
			1.2 & Napisane upute za puštanje u pogon & V. Kumanović & 8.12.2023. \\[3pt] \hline
			1.3 & Dijagram aktivnosti & L. Crvelin & 13.12.2023. \\[3pt] \hline
			1.4 & Dijagram razmještaja & V. Kumanović & 14.12.2023. \\[3pt] \hline
			1.5 & Korišteni alati i tehnologije & L. Crvelin & 18.12.2023. \\[3pt] \hline
			1.6 & Zaključak i budući rad & L. Crvelin & 21.12.2023. \\[3pt] \hline
			1.7 & Izmjena funkcionalnih zahtjeva i opisa obraza uporabe & L. Crvelin & 21.12.2023. \\[3pt] \hline
			1.7.1 & Izmjena UC dijagrama, sekvencijskih dijagrama i dijagrama aktivnosti & L. Crvelin & 23.12.2023. \\[3pt] \hline
			1.8 & Dijagram stanja i komponenti & V. Kumanović & 28.12.2023. \\[3pt] \hline
			1.9 & Dokumentacija testiranja & D. Matić & 12.1.2024. \\[3pt] \hline

			% 0.8 & Povijest rada i trenutni status implementacije,\newline Zaključci i plan daljnjeg rada & * & dd.mm.2023. \\[3pt] \hline 
			% 0.9 & Opisi obrazaca uporabe & * & dd.mm.2023. \\[3pt] \hline 
			% 0.10 & Preveden uvod & * & dd.mm.2023. \\[3pt] \hline 
			% 0.11 & Sekvencijski dijagrami & * & dd.mm.2023. \\[3pt] \hline 
			% 0.12.1 & Započeo dijagrame razreda & * & dd.mm.2023. \\[3pt] \hline 
			% 0.12.2 & Nastavak dijagrama razreda & * & dd.mm.2023. \\[3pt] \hline 
			% \textbf{1.0} & Verzija samo s bitnim dijelovima za 1. ciklus & * & dd.mm.2023. \\[3pt] \hline 
			% 1.1 & Uređivanje teksta -- funkcionalni i nefunkcionalni zahtjevi & * \newline * & dd.mm.2023. \\[3pt] \hline 
			% 1.2 & Manje izmjene:Timer - Brojilo vremena & * & dd.mm.2023. \\[3pt] \hline 
			% 1.3 & Popravljeni dijagrami obrazaca uporabe & * & dd.mm.2023. \\[3pt] \hline 
			% 1.5 & Generalna revizija strukture dokumenta & * & dd.mm.2023. \\[3pt] \hline 
			% 1.5.1 & Manja revizija (dijagram razmještaja) & * & dd.mm.2023. \\[3pt] \hline 
			% \textbf{2.0} & Konačni tekst predloška dokumentacije  & * & dd.mm.2023. \\[3pt] \hline 
			% &  &  & \\[3pt] \hline	
		\end{longtblr}
	
		%Moraju postojati glavne revizije dokumenata 1.0 i 2.0 na kraju prvog i drugog ciklusa. Između tih revizija mogu postojati manje revizije 0.1, 0.2, ..., 1.0., 1.1, 1.2, ..., 2.0 
	\chapter{Opis projektnog zadatka}
		
		%dio 1. revizije
        %min 3, optimalno 4-5 str opisa
        %teme:
        %-potencijalna korist ovog projekta
		%-postojeća slična rješenja (istražiti i ukratko opisati razlike u odnosu na zadani zadatak). Dodajte slike koja predočavaju slična rješenja.
		%-skup korisnika koji bi mogao biti zainteresiran za ostvareno rješenje.
		%-mogućnost prilagodbe rješenja 
		%-opseg projektnog zadatka
		%-moguće nadogradnje projektnog zadatka
            
		
		\textit{}
		\eject
		
		
	
	\chapter{Specifikacija programske potpore}
		
	\section{Funkcionalni zahtjevi}
			
			\noindent \textbf{Dionici:}
			
			\begin{packed_enum}
				
				\item Roditelj
				\item Pedijatar
                \item Liječnik obiteljske medicine
                \item Administrator
                
			\end{packed_enum}
			
			\noindent \textbf{Aktori i njihovi funkcionalni zahtjevi:}
			
			
			\begin{packed_enum}
				\item  \underbar{Roditelj (inicijator) može:}
				
				\begin{packed_enum}
    
					\item registrirati se i prijaviti u sustav
                    \item vidjeti profil za sebe i svoju djecu
					\item pregledati medicinske nalaze, dijagnoze i povijesti pregleda za sebe i svoju djecu
                    \item zatražiti povratnu informaciju liječnika ili pedijatra o nalazu dobivenim temeljem određene usluge u privatnoj ustanovi, tj. drugog mišljenja
                    \item vidjeti druga mišljenja koja je stvorio
                    
				\end{packed_enum}
			
				\item  \underbar{Pedijatar (inicijator) može:}
				
				\begin{packed_enum}
			
					\item prijaviti se u sustav
                    \item prijaviti djecu po identifikatoru (OIB)
                    \item pregledati popis sve prijavljene djece i njihove kartone
					\item unijeti medicinske podatke (dijagnoze, preglede, terapije) i ažurirati te podatke
                    \item evidentirati preporuku za bolovanje za roditelja u slučaju bolesti djeteta
                    \item utvrditi bolest djeteta nakon čega se šalje ispričnica u školu ili vrtić
                    \item naručiti djecu na specijalističke preglede/postupke
                    \item dati povratnu informaciju na drugo mišljenje
                
				\end{packed_enum}

                \item  \underbar{Liječnik obiteljske medicine (inicijator) može:}
				
				\begin{packed_enum}
			
					\item prijaviti se u sustav
                    \item pregledati i odobriti doznake izdane od strane pedijatra koje se šalju poslodavcu roditelja
                    \item dati povratnu informaciju na drugo mišljenje
                    \item naručiti pacijenta na specijalistički pregled/postupak
                    \item unositi podatke o pregledima
                    \item pristupiti medicinskim nalazima, povijesti pregleda i upitima 
                    
				\end{packed_enum}
    
                \item  \underbar{Administrator (inicijator) može:}
				
				\begin{packed_enum}

                    \item prijava u sustav kao administrator
					\item pripremiti registre djece (osnovni podaci, OIB)
                    \item registrira liječnika obiteljske medicine i pedijatra
                    \item po registraciji roditelja, povezati iste s djecom (preko OIB-a)
					\item administrirati korisničke račune
                    \item ažurirati sve podatke u aplikaciji
                 
				\end{packed_enum}

                \item  \underbar{Baza podataka (sudionik):}
				
				\begin{packed_enum}

                    \item pohranjuje sve podatke o korisnicima i njihovim ovlastima
					
				\end{packed_enum}
    
			\end{packed_enum}
			
			\eject 
			
			
				
			\subsection{Obrasci uporabe}
				
				%dio 1. revizije
				
				\subsubsection{Opis obrazaca uporabe}
					%Funkcionalne zahtjeve razraditi u obliku obrazaca uporabe. Ukoliko u nekom koraku može doći do odstupanja, potrebno je to odstupanje opisati i po mogućnosti ponuditi rješenje kojim bi se tijek obrasca vratio na osnovni tijek.
					

					\noindent \underbar{\textbf{UC$<$broj obrasca$>$ -$<$ime obrasca$>$}}
					\begin{packed_item}
	
						\item \textbf{Glavni sudionik: }$<$sudionik$>$
						\item  \textbf{Cilj:} $<$cilj$>$
						\item  \textbf{Sudionici:} $<$sudionici$>$
						\item  \textbf{Preduvjet:} $<$preduvjet$>$
						\item  \textbf{Opis osnovnog tijeka:}
						
						\item[] \begin{packed_enum}
	
							\item $<$opis korak jedan$>$
							\item $<$opis korak dva$>$
							\item $<$opis korak tri$>$
							\item $<$opis korak četiri$>$
							\item $<$opis korak pet$>$
						\end{packed_enum}
						
						\item  \textbf{Opis mogućih odstupanja:}
						
						\item[] \begin{packed_item}
	
							\item[2.a] $<$opis mogućeg scenarija odstupanja u koraku 2$>$
							\item[] \begin{packed_enum}
								
								\item $<$opis rješenja mogućeg scenarija korak 1$>$
								\item $<$opis rješenja mogućeg scenarija korak 2$>$
								
							\end{packed_enum}
							\item[2.b] $<$opis mogućeg scenarija odstupanja u koraku 2$>$
							\item[3.a] $<$opis mogućeg scenarija odstupanja  u koraku 3$>$
							
						\end{packed_item}
					\end{packed_item}
				
					
				\subsubsection{Dijagrami obrazaca uporabe}
					
					%Prikazati odnos aktora i obrazaca uporabe odgovarajućim UML dijagramom. Nije nužno nacrtati sve na jednom dijagramu. Modelirati po razinama apstrakcije i skupovima srodnih funkcionalnosti.
				\eject		
				
			\subsection{Sekvencijski dijagrami}
				
				%dio 1. revizije
				
				%Nacrtati sekvencijske dijagrame koji modeliraju najvažnije dijelove sustava (max. 4 dijagrama). Uz svaki dijagram napisati detaljni opis dijagrama.
    
				\eject
	
		\section{Ostali zahtjevi}
		
			%dio 1. revizije
		 
			 %Nefunkcionalni zahtjevi i zahtjevi domene primjene dopunjuju funkcionalne zahtjeve. Oni opisuju kako se sustav treba ponašati i koja ograničenja treba poštivati (performanse, korisničko iskustvo, pouzdanost, standardi kvalitete, sigurnost...). Primjeri takvih zahtjeva u Vašem projektu mogu biti: podržani jezici korisničkog sučelja, vrijeme odziva, najveći mogući podržani broj korisnika, podržane web/mobilne platforme, razina zaštite (protokoli komunikacije, kriptiranje...)... Svaki takav zahtjev potrebno je navesti u jednoj ili dvije rečenice.
			 
			 
			 
	
	\chapter{Arhitektura i dizajn sustava}
		
		%dio 1. revizije

		%Potrebno je opisati stil arhitekture te identificirati: podsustave, preslikavanje na radnu platformu, spremišta podataka, mrežne protokole, globalni upravljački tok i sklopovsko-programske zahtjeve. Po točkama razraditi i popratiti odgovarajućim skicama:
	% - izbor arhitekture temeljem principa oblikovanja pokazanih na           predavanjima (objasniti zašto ste baš odabrali takvu arhitekturu)
	% - organizaciju sustava s najviše razine apstrakcije (npr. klijent-       poslužitelj, baza podataka, datotečni sustav, grafičko sučelje)
	% - organizaciju aplikacije (npr. slojevi frontend i backend, MVC          arhitektura)		
	

					
		\section{Baza podataka}
			
			%dio 1. revizije
			
		%Potrebno je opisati koju vrstu i implementaciju baze podataka ste odabrali, glavne komponente od kojih se sastoji i slično.
        Baza podataka koja se koristi u projektu "Ozdravi" ključna je komponenta sustava koja omogućava pohranu, upravljanje i praćenje svih relevantnih informacija o korisnicima i njihovim zdravstvenim podacima. Odabrana je relacijska baza podataka koja se sastoji od nekoliko ključnih entiteta:
            \begin{packed_item}
                \item Person
                \item Examination
                \item SecondOpinion
                \item Instruction
                \item SickLeaveRecommendation
                \item LoginCreds
            \end{packed_item}

        Baza podataka "Ozdravi" omogućava integraciju svih informacija kako bi podržala procese komunikacije, pregleda i dijagnoza u kontekstu brige o zdravstvenim potrebama djece. Ova struktura omogućava učinkovito praćenje i upravljanje svim relevantnim medicinskim podacima i podržava glavne funkcionalnosti sustava.
        
			\subsection{Opis tablica}

            \textbf{Osoba} Ovaj entitet sadržava osnovne informacije o svim korisnicima sustava, uključujući njihovu ulogu, identifikacijske brojeve, ime, prezime, korisničko ime i lozinku. Ova tablica je ključna za autentikaciju korisnika i definira njihove uloge unutar sustava.
				\begin{longtblr}[
					label=none,
					entry=none
					]{
						width = \textwidth,
						colspec={|X[6,l]|X[6, l]|X[20, l]|}, 
						rowhead = 1,
					} %definicija širine tablice, širine stupaca, poravnanje i broja redaka naslova tablice
					\hline \SetCell[c=3]{c}{\textbf{Person (osoba)}}	 \\ \hline[3pt]
					\SetCell{LightGreen}id & INT	&  	- 	\\ \hline
					\SetCell{LightBlue}parent\_id	& INT & ID Osobe koja je roditelj osobe [Parent] \\ \hline 
                    \SetCell{LightBlue}doctor\_id	& INT & ID Osobe koja je doktor osobe [Doctor/Pediatrician] \\ \hline 
                    \SetCell{LightBlue}address\_id	& INT & ID Adrese osobe \\ \hline 
					OIB & VARCHAR & Personal ID Number \\ \hline 
					first\_name & VARCHAR &  Ime osobe (NOT NULL) \\ \hline 
                    last\_name & VARCHAR &  Prezime osobe (NOT NULL) \\ \hline 
                    role & VARCHAR &  Uloga korisnika (NOT NULL) \\ \hline 
                    password & VARCHAR &  Lozinka za prijavu u sustav \\ \hline
                    email & VARCHAR &  e-mail za prijavu u sustav \\ \hline 
					
				\end{longtblr}

                \textbf{Pregled} Ovaj entitet sadržava podatke o pacijentima, uključujući informacije o njihovim pregledima, dijagnozama i zapisnicima. Sadrži veze između pacijenata, doktora i pregleda, omogućavajući praćenje medicinskih podataka.
                
                \begin{longtblr}[
					label=none,
					entry=none
					]{
						width = \textwidth,
						colspec={|X[6,l]|X[6, l]|X[20, l]|}, 
						rowhead = 1,
					} 
					\hline \SetCell[c=3]{c}{\textbf{Examination (pregled)}}	 \\ \hline[3pt]
					\SetCell{LightGreen}id & INT	&  	- 	\\ \hline
					\SetCell{LightBlue}patient\_id	& INT & ID Osobe koja je pacijent [Parent/Child] (NOT NULL) \\ \hline 
                    \SetCell{LightBlue}doctor\_id	& INT & ID Osobe koja vrši pregled [Doctor/Pediatrician] (NOT NULL) \\ \hline 
                    \SetCell{LightBlue}scheduler\_id	& INT & ID osobe koja je zakazala pregled [Doctor/Pediatriciran] (NOT NULL) \\ \hline 
                    \SetCell{LightBlue}address\_id	& INT & ID adrese na kojoj se održava pregled \\ \hline 
					report & TEXT & Dijagnoza/zapisnik s pregleda/... \\ \hline 
					date & DATETIME &  Datum i vrijeme pregleda (NOT NULL) \\ \hline 
				\end{longtblr}

                \textbf{Drugo mišljenje} Ovaj entitet služi za praćenje detalja o pregledima pacijenata. Sadrži informacije o osobama koje su zatražile drugo mišljenje, doktorima koji su dali mišljenje te samom sadržaju mišljenja. To omogućava praćenje medicinskih konzultacija.
                
                 \begin{longtblr}[
					label=none,
					entry=none
					]{
						width = \textwidth,
						colspec={|X[6,l]|X[6, l]|X[20, l]|}, 
						rowhead = 1,
					} 
					\hline \SetCell[c=3]{c}{\textbf{SecondOpinion (DrugoMišljenje)}}	 \\ \hline[3pt]
					\SetCell{LightGreen}id & INT	&  	- 	\\ \hline
					\SetCell{LightBlue}requester\_id	& INT & ID Osobe koja je zatražila drugo mišljenje [Parent] (NOT NULL) \\ \hline 
                    \SetCell{LightBlue}doctor\_id & INT & ID Osobe koja je dala drugo mišljenje [Doctor/Pediatrician] (NOT NULL) \\ \hline 
					second_opinion & TEXT & Drugo mišljenje doktora (NOT NULL)\\ \hline 
					content & TEXT &  Sadržaj na koji se daje drugo mišljenje (nalaz iz privatne ustanove) \\ \hline 
				\end{longtblr}

                \textbf{Uputa} Ovaj entitet sadržava informacije o izdanim uputama, uključujući informacije o pacijentima, doktorima koji stvaraju upute i statusu upute. Omogućava praćenje i upravljanje medicinskim preporukama.
                
                \begin{longtblr}[
					label=none,
					entry=none
					]{
						width = \textwidth,
						colspec={|X[6,l]|X[6, l]|X[20, l]|}, 
						rowhead = 1,
					} 
					\hline \SetCell[c=3]{c}{\textbf{Instruction (uputa)}}	 \\ \hline[3pt]
					\SetCell{LightGreen}id & INT	&  	- 	\\ \hline
					\SetCell{LightBlue}doctor\_id	& INT & ID Osobe koja je izdala uputu [Doctor/Pediatrician] (NOT NULL) \\ \hline 
                    \SetCell{LightBlue}patient\_id & INT & ID Osobe za koju je uputa [Parent] (NOT NULL) \\ \hline 
					date & DATETIME & Datum i vrijeme izdavanja upute (NOT NULL)\\ \hline 
					content & TEXT &  Sadržaj uput (NOT NULL) \\ \hline 
				\end{longtblr}

                \textbf{Preopurka za bolovanje} Ovaj enitet sadrži informacije o izdanim preporukama za bolovanje.  
                \begin{longtblr}[
					label=none,
					entry=none
					]{
						width = \textwidth,
						colspec={|X[6,l]|X[6, l]|X[20, l]|}, 
						rowhead = 1,
					} 
					\hline \SetCell[c=3]{c}{\textbf{SickLeaveRecommendation (Preporuka za bolovanje)}}	 \\ \hline[3pt]
					\SetCell{LightGreen}id & INT	&  	- 	\\ \hline
                    \SetCell{LightBlue}patient\_id & INT & ID Osobe za koju se traži bolovanje [Parent] (NOT NULL) \\ \hline 
                    \SetCell{LightBlue}creator\_id & INT & ID Osobe koja je stvorila preporuku [Pediatrician] (NOT NULL) \\ \hline
                    \SetCell{LightBlue}approver\_id & INT & ID Osobe koja mora odobriti preporuku [Doctor] (NOT NULL) \\ \hline 
					\SetCell{LightBlue}examination\_id	& INT & ID pregleda na temelju kojeg se izdaje preporuka (NOT NULL) \\ \hline 
					email & VARCHAR & e-mail na koji se šalje, ako je potvrđena (NOT NULL) \\ \hline 
					status & BOOLEAN &  Je li preporuka odobrena \\ \hline 
				\end{longtblr}

			\subsection{Dijagram baze podataka}
				% U ovom potpoglavlju potrebno je umetnuti dijagram baze podataka. Primarni i strani ključevi moraju biti označeni, a tablice povezane. Bazu podataka je potrebno normalizirati.
			
			\eject
			
			
		\section{Dijagram razreda}
		
			%Potrebno je priložiti dijagram razreda s pripadajućim opisom. Zbog preglednosti je moguće dijagram razlomiti na više njih, ali moraju biti grupirani prema sličnim razinama apstrakcije i srodnim funkcionalnostima.
			
			%dio 1. revizije
			
			%Prilikom prve predaje projekta, potrebno je priložiti potpuno razrađen dijagram razreda vezan uz generičku funkcionalnost sustava. Ostale funkcionalnosti trebaju biti idejno razrađene u dijagramu sa sljedećim komponentama: nazivi razreda, nazivi metoda i vrste pristupa metodama (npr. javni, zaštićeni), nazivi atributa razreda, veze i odnosi između razreda.
			
			%dio 2. revizije			
			
			%Prilikom druge predaje projekta dijagram razreda i opisi moraju odgovarati stvarnom stanju implementacije
			
			
			
			\eject
		
		\section{Dijagram stanja}
			
			
			%dio 2. revizije
			
			%Potrebno je priložiti dijagram stanja i opisati ga. Dovoljan je jedan dijagram stanja koji prikazuje značajan dio funkcionalnosti sustava. Na primjer, stanja korisničkog sučelja i tijek korištenja neke ključne funkcionalnosti jesu značajan dio sustava, a registracija i prijava nisu.
			
			
			\eject 
		
		\section{Dijagram aktivnosti}
			
			%dio 2. revizije
			
			 %Potrebno je priložiti dijagram aktivnosti s pripadajućim opisom. Dijagram aktivnosti treba prikazivati značajan dio sustava.
			
			\eject
		\section{Dijagram komponenti}
		
			%dio 2. revizije
		
			 %Potrebno je priložiti dijagram komponenti s pripadajućim opisom. Dijagram komponenti treba prikazivati strukturu cijele aplikacije.
	\chapter{Implementacija i korisničko sučelje}
		\section{Korištene tehnologije i alati}		
			 U tijeku razvoja projekta "Ozdravi", pažljivo su odabrane i implementirane različite tehnologije i alati kako bi se postigla efikasna izrada dokumentacije i aplikacije za zdravstveni sustav.
			 Za unapređenje timskih komunikacija korištene su aplikacije \textit{WhatsApp}\footnote{\url{https://www.whatsapp.com}} i \textit{Microsoft Teams}\footnote{\url{https://www.microsoft.com/en-us/microsoft-teams/group-chat-software}}. WhatsApp, popularna platforma za brzu razmjenu poruka, olakšala je neposrednu interakciju među članovima tima, dok je Microsoft Teams pružio šire mogućnosti za suradnju, uključujući organizaciju sastanaka, održavanje sastanaka i dijeljenje datoteka.
			 U praćenju zadataka i upravljanju projektom koristila se \textit{Jira}\footnote{\url{https://www.atlassian.com/software/jira}}, snažan alat koji je omogućio stvaranje strukturirane i pregledne radne okoline. Jira je poslužila kao središnje mjesto za praćenje i planiranje aktivnosti, olakšavajući koordinaciju među članovima tima.
			 \textit{GitHub}\footnote{\url{https://github.com}} se istaknuo kao ključni alat za upravljanje izvornim kodom i omogućavanje učinkovite suradnje među članovima tima. GitHub, kao web-bazirana platforma, pružio je mogućnost distribuiranog upravljanja verzijama korištenjem Git sustava. Repozitorij na GitHubu služio je kao centralno mjesto za čitav izvorni kod, omogućavajući članovima tima sinkroniziranje svojih lokalnih kopija koda, praćenje promjena te učinkovito upravljanje raznim granama razvoja.
			 \textit{Astah}\footnote{\url{http://astah.net/editions/professional}} je korišten za izradu UML dijagrama, pružajući jasan prikaz strukture sustava i odnosa između njegovih dijelova. Osim Astaha, korišteni su i drugi alati za izradu dijagrama, uključujući \textit{Visual Paradigm}\footnote{\url{https://online.visual-paradigm.com}} i \textit{Lucidchart}\footnote{\url{https://www.lucidchart.com}}.
			 Za razvoj frontend dijela aplikacije korišten je \textit{React}\footnote{\url{https://reactjs.org/}}, jedna od najpopularnijih \textit{JavaScript}\footnote{\url{https://www.javascript.com/}} biblioteka koja omogućava dinamičko i responzivno korisničko sučelje. React je pridonosio brzoj izgradnji modernog sučelja s visokom interaktivnošću.
			 Korišten je i \textit{Bootstrap}\footnote{\url{https://getbootstrap.com}}, popularni okvir za izradu responzivnih web stranica, koji je omogućio brzo i jednostavno oblikovanje sučelja te moderan i funkcionalan dizajn. Bootstrap pruža jednostavne i učinkovite klase za organizaciju elemenata, responzivno oblikovanje pomoću tzv. grid sustava te unaprijed definirane stilove za tipografiju, forme, gumbiće i mnoge druge komponente.
			 Za backend je odabran \textit{Spring}\footnote{\url{https://spring.io}}, robustan okvir za razvoj \textit{Java}\footnote{\url{https://www.java.com/en/}} aplikacija. Spring je pružio strukturu i podršku za izradu pouzdanih, sigurnih i skalabilnih backend rješenja.
			 API sustav je strukturiran i dokumentiran korištenjem alata \textit{Swagger}\footnote{\url{https://swagger.io}}, pri čemu je implementiran sukladno OpenAPI konvencijama. OpenAPI pristup omogućava jednostavno definiranje, dokumentiranje i testiranje RESTful API-ja te definiranje svih dostupnih endpointa, parametara, odgovora i modela podataka.
			 U procesu pisanja koda korišteni su \textit{IntelliJ IDEA}\footnote{\url{https://www.jetbrains.com/idea/}} i \textit{Visual Studio Code}\footnote{\url{https://visualstudio.microsoft.com/}}. IntelliJ IDEA, s bogatim skupom značajki, podržavao je razvoj Java aplikacija, dok je Visual Studio Code, lagan i svestran, omogućavao efikasno pisanje koda u različitim jezicima, uključujući \textit{JavaScript}\footnote{\url{https://www.javascript.com/}}.
			 Odabir ovih alata bio je prepušten osobnim preferencijama članova tima.
			 Za postavljanje produkcijskog okruženja odabran je Heroku kao infrastrukturna platforma. \textit{Heroku}\footnote{\url{https://www.heroku.com}} pruža stabilnost i skalabilnost, čineći implementaciju i održavanje aplikacije jednostavnim i učinkovitim.
			 Baza podataka \textit{PostgreSQL}\footnote{\url{https://www.postgresql.org}} koristila se za trajno pohranjivanje podataka. PostgreSQL je open-source sustav za upravljanje bazama podataka koji pruža pouzdano i sigurno okruženje za pohranu informacija.
			 U konačnici, ovom kombinacijom tehnologija postignuto je stvaranje moderne zdravstvene aplikacije koja je intuitivna za korisnike, prilagodljiva potrebama zdravstvenog sustava i unapređuje ukupno iskustvo krajnjih korisnika.
			\eject 
	
		\section{Ispitivanje programskog rješenja}
			Ispitivanje programskog rješenja predstavlja ključan korak u procesu razvoja programske podrške s ciljem ostvarivanja nekoliko ključnih prednosti:
			\begin{packed_item}
				\item olakšava dodavanje značajki - ispitivanje osigurava da nove promjene ne narušavaju funkcionalnost postojećeg koda
				\item sprječava unošenje grešaka u kod - kvalitetni testovi detektiraju neispravan kod prije nego što postane dio projekta čime se smanjuje mogućnost pojave grešaka
			\end{packed_item}
			
			U sklopu ovog projekta koristili smo dvije vrste ispitivanja:
			\begin{packed_item}
				\item Ispitivanje komponenti (eng. Unit Testing)
				\item Ispitivanje sustava (eng. Integration Testing)
			\end{packed_item}
			
			
			Ispitivanjem komponenti koristeći JUnit biblioteke osigurali smo ispravan rad ključnih dijelovi backenda. Ispitivanjem sustava provjerili smo ispravnost povezanosti backenda i frontenda iz perspektive krajnjeg korisnika.
			Ispitivanje sustava je provedeno koristeći Selenium biblioteke.
			
			%==========================================================================================================================%
			\subsection{Ispitivanje komponenti}
			\subsubsection*{Test 1: Ispitivanje ispravnog OIB-a}
			Komponenta: ValidityUtil \newline
			Metoda: public static boolean isValidOib(String oib); \newline
			Ulazni Podaci: 
			\begin{packed_item}
				\item String oib : "39751670659"
			\end{packed_item}
			Očekivani rezultat:
			\begin{packed_item}
				\item Metoda vraća istinu (true) jer je OIB ispravan prema standardu za OIB-e.
			\end{packed_item}
			\lstinputlisting[style=JavaStyle, caption={Test 1}]{code/oib_test_1.java}

			\subsubsection*{Test 2: Ispitivanje neispravnog OIB-a}
			Komponenta: ValidityUtil \newline
			Metoda: public static boolean isValidOib(String oib); \newline
			Ulazni Podaci: 
			\begin{packed_item}
				\item String oib : "397516706.9"
			\end{packed_item}
			Očekivani rezultat:
			\begin{packed_item}
				\item Metoda vraća laž (false) jer OIB sadrži '.', odnosno ne zadovoljava standardni oblik.
			\end{packed_item}
			\lstinputlisting[style=JavaStyle, caption={Test 2}]{code/oib_test_2.java}

			U testovima namijenjenima testiranju kontrolera, želimo ispitati isključivo ponašanje kontrolera, bez ispitivanja servisa, baze i drugih dijelova aplikacije.
			Budući da kontroleri ne mogu raditi bez tih dijelova, nužno je stvoriti "mock" objekte koji ih mijenjaju i simuliraju njihovo ponašanje.

			Ovaj postupak uključuje ručno stvaranje entiteta koji predstavljaju očekivane rezultate servisa.

			\subsubsection*{Test 3: Dohvaćanje popisa Uputa.}
			Komponenta: InstructionsController \newline
			Zahtjev: HTTP GET /instructions \newline
			Mock Podaci:
			\begin{packed_item}
				\item Korisnik
				\item Pregledi kojima korisnik može pristupiti
			\end{packed_item}
			Očekivani rezultat:
			\begin{packed_item}
				\item Kontroler u JSON formatu vraća popis svih Uputa koje trenutni korisnik može vidjeti, HTTP odgovor ima status 201 (OK).
			\end{packed_item}
			\lstinputlisting[style=JavaStyle, caption={Test 3}]{code/instruction_test_1.java}
			
			\subsubsection*{Test 4: Stvaranje Uputa.}
			Komponenta: InstructionsController \newline
			Zahtjev: HTTP POST /instructions \newline
			Mock Podaci:
			\begin{packed_item}
				\item Korisnik (ADMIN)
			\end{packed_item}
			Očekivani rezultat:
			\begin{packed_item}
				\item Kontroler u JSON formatu vraća atribute novostvorene Upute, HTTP odgovor ima status 201 (Created).
			\end{packed_item}
			\lstinputlisting[style=JavaStyle, caption={Test 4}]{code/instruction_test_2.java}

			\subsubsection*{Test 5: Dohvaćanje nepostojeće Upute.}
			Komponenta: InstructionsController \newline
			Zahtjev: HTTP GET /instruction/\{id\} \newline
			Mock Podaci:
			\begin{packed_item}
				\item Korisnik (ADMIN)
			\end{packed_item}
			Očekivani rezultat:
			\begin{packed_item}
				\item HTTP odgovor ima status 404 (Not Found).
			\end{packed_item}
			\lstinputlisting[style=JavaStyle, caption={Test 5}]{code/instruction_test_3.java}

			\subsubsection*{Test 6: Dohvaćanje svih doktora u aplikaciji.}
			Komponenta: UsersController \newline
			Zahtjev: HTTP GET /users/doctors \newline
			Mock Podaci:
			\begin{packed_item}
				\item Korisnik (PARENT)
			\end{packed_item}
			Očekivani rezultat:
			\begin{packed_item}
				\item HTTP odgovor ima status 403 (Forbidden), jer roditelj ne smije dobiti popis svih liječnika u aplikaciji.
			\end{packed_item}
			\lstinputlisting[style=JavaStyle, caption={Test 6}]{code/user_test_1.java}
			
			\subsubsection*{Test 7: Dodavanje korisnika.}
			Komponenta: UsersController \newline
			Zahtjev: HTTP POST /users \newline
			Mock Podaci:
			\begin{packed_item}
				\item Korisnik (ADMIN)
			\end{packed_item}
			Očekivani rezultat:
			\begin{packed_item}
				\item Kontroler u JSON formatu vraća atribute novostvorenog Korisnika, HTTP odgovor ima status 201 (Created).
			\end{packed_item}
			\lstinputlisting[style=JavaStyle, caption={Test 7}]{code/user_test_2.java}

			\subsubsection*{Test 8: Registracija korisnika.}
			Komponenta: AuthenticationController \newline
			Zahtjev: HTTP POST /register \newline
			Očekivani rezultat:
			\begin{packed_item}
				\item Kontroler u JSON formatu vraća atribute novostvorenog Korisnika, HTTP odgovor ima status 201 (Created).
			\end{packed_item}
			\lstinputlisting[style=JavaStyle, caption={Test 8}]{code/registration_test.java}

			\begin{figure}[H]
				\includegraphics[width=\textwidth]{slike/testing_results.png} 
				\caption{Rezultati testiranja na lokalnom stroju.} 
			\end{figure}
			
			%==========================================================================================================================%
			\subsection{Ispitivanje sustava}
			Testovi ispitivanja sustava su napisani u Selenium IDE dodatku za preglednik i prate nekoliko obrazaca uporabe, ali su vezani uz jednu cjelinu - rad s preporukama za bolovanje.
			U ovom scenariju testiranja sudjeluje više uloga (roditelj, pedijatar, liječnik) i svi koraci koje jedna uloga izvodi su grupirani u zasebnu grupu koraka. Također, za svaki korak je
			naveden i pripadajući očekivani rezultat.
			
			 \subsubsection*{Grupa koraka 1: Dodavanje preporuke za bolovanje}
			 Koraci:
			 \begin{packed_enum}
				\item Prijava u sustav kao pedijatar ("pedijatar@mail.com")
				\item Odabir opcije "Bolovanja" iz alatne trake.
				\item Odabir opcije "Dodaj preporuku"
				\item Odabir pregleda na temelju kojeg se izdaje preporuka ("Tamara Stanic 2023-12-25") unutar modalnog ekrana.
				\item Odabir opcije "Spremi".
			 \end{packed_enum}
			 Očekivani rezultati:
			 \begin{packed_enum}
				\item Pedijatar se može prijaviti u sustav.
				\item Postoji opcija "Bolovanja" u alatnoj traci.
				\item Dostupna je opcija "Dodaj preporuku".
				\item Postoji izbornik pregleda unutar modalnog ekrana u kojem je moguće izabrati traženi pregled.
				\item Postoji opcija "Spremi" unutar modalnog ekrana.
			 \end{packed_enum}
			 
			 \begin{figure}[H]
				\includegraphics[width=\textwidth]{slike/selenium1.1.png} 
				\caption{Rezultati testiranja grupe koraka 1} 
			\end{figure}

			\begin{figure}[H]
				\includegraphics[width=\textwidth]{slike/selenium1.2.png} 
				\caption{Rezultati testiranja grupe koraka 1} 
			\end{figure}

			 \subsubsection*{Grupa koraka 2: Pregled neobrađene preporuke za bolovanje}
			 Koraci:
			 \begin{packed_enum}
				\item Prijava u sustav kao roditelj ("roditelj@mail.com")
				\item Odabir opcije "Bolovanja" iz alatne trake.
				\item Odabir preporuke iz popisa preporuke ("Tamara Stanic 2023-12-25").
			 \end{packed_enum}
			 Očekivani rezultati:
			 \begin{packed_enum}
				\item Roditelj se može prijaviti u sustav.
				\item Postoji opcija "Bolovanja" u alatnoj traci.
				\item Postoji tražena preporuka za bolovanje u popisu preporuka.
				\item Pod "Status" piše "Čeka odobrenje." unutar modalnog ekrana.
			 \end{packed_enum}

			 
			 \begin{figure}[H]
				\includegraphics[width=\textwidth]{slike/selenium2.1.png} 
				\caption{Rezultati testiranja grupe koraka 2} 
			\end{figure}

			\begin{figure}[H]
				\includegraphics[width=\textwidth]{slike/selenium2.2.png} 
				\caption{Rezultati testiranja grupe koraka 2} 
			\end{figure}

			 \subsubsection*{Grupa koraka 3: Odobravanje preporuke za bolovanje}
			 Koraci:
			 \begin{packed_enum}
				\item Prijava u sustav kao liječnik obiteljske medicine ("doktor@mail.com")
				\item Odabir opcije "Bolovanja" iz alatne trake.
				\item Odabir preporuke iz popisa preporuke ("Tamara Stanic 2023-12-25").
				\item Odabir opcije "Odobri" iz modalnog ekrana.
			 \end{packed_enum}
			 Očekivani rezultati:
			 \begin{packed_enum}
				\item Liječnik obiteljske medicine se može prijaviti u sustav.
				\item Postoji opcija "Bolovanja" u alatnoj traci.
				\item Postoji tražena preporuka za bolovanje u popisu preporuka.
				\item Postoji opcija "Odobri" unutar modalnog okvira.
			 \end{packed_enum}

			 \begin{figure}[H]
				\includegraphics[width=\textwidth]{slike/selenium3.1.png} 
				\caption{Rezultati testiranja grupe koraka 3} 
			\end{figure}

			\begin{figure}[H]
				\includegraphics[width=\textwidth]{slike/selenium3.2.png} 
				\caption{Rezultati testiranja grupe koraka 3} 
			\end{figure}

			 \subsubsection*{Grupa koraka 4: Pregled odobrene preporuke za bolovanje}
			 Koraci:
			 \begin{packed_enum}
				\item Prijava u sustav kao roditelj ("roditelj@mail.com")
				\item Odabir opcije "Bolovanja" iz alatne trake.
				\item Odabir preporuke iz popisa preporuke ("Tamara Stanic 2023-12-25").
			 \end{packed_enum}
			 Očekivani rezultati:
			 \begin{packed_enum}
				\item Roditelj se može prijaviti u sustav.
				\item Postoji opcija "Bolovanja" u alatnoj traci.
				\item Postoji tražena preporuka za bolovanje u popisu preporuka.
				\item Pod "Status" piše "Odobreno." u modalnom ekranu preporuke.
			 \end{packed_enum}
			
			\begin{figure}[H]
				\includegraphics[width=\textwidth]{slike/selenium4.1.png} 
				\caption{Rezultati testiranja grupe koraka 4} 
			\end{figure}

			\begin{figure}[H]
				\includegraphics[width=\textwidth]{slike/selenium4.2.png} 
				\caption{Rezultati testiranja grupe koraka 4} 
			\end{figure}

			\eject
		
		\section{Dijagram razmještaja}
			Dijagrami razmještaja prikazuju konfiguraciju sklopovlja i programske podrške koja se koristi za implementaciju sustava u njegovom operativnom okruženju. Na cloud platformi smješten je poslužitelj baze podataka te web poslužitelji za backend i frontend. Korisnici pristupaju web aplikaciji putem svog web preglednika. Sustav je temeljen na arhitekturi "klijent - poslužitelj" s komunikacijom između korisničkih računala (roditelj, pedijatar, doktor, administrator) i poslužitelja putem HTTPS veze.
			\begin{figure}[H]
				\includegraphics[width=\textwidth]{slike/deploymentDiagram.png} 
				\caption{Dijagram razmještaja} 
			\end{figure}
			\eject 
		
		\section{Upute za puštanje u pogon}
			Aplikacija "Ozdravi" se sastoji od nekoliko dijelova koje je potrebno postaviti kako bi se pokrenula u lokalnom okruženju.
		
			\subsection*{Postavljanje baze podataka}
			U procesu postavljanja baze podataka za aplikaciju "Ozdravi", prvi korak je instalacija relacijske baze podataka
			\href{https://www.postgresql.org/download/}{\textit{PostgreSQL-a}}. Nakon završetka instalacije, slijedi standardni postupak postavljanja korisničkog imena i lozinke (neobavezno).
			Nakon uspješne instalacije, potrebno je pristupiti konzoli baze (psql) kako bi se stvorili potrebni resursi za rad backenda aplikacije "Ozdravi".
	
			Nakon toga, izvršavaju se sljedeće naredbe:
			\lstinputlisting[style=SQLStyle, caption={Postavljanje baze podataka}]{code/db_setup.sql}

			Sada bi trebala biti stvorena baza podataka "ozdravi", čiji je vlasnik korisnik "welebyte". Naredbom \textit{\textbackslash l} može se provjeriti lista svih postojećih 
			baza podataka i njihovih vlasnika.
			\begin{figure}[H]
				\includegraphics[width=\textwidth]{slike/sqlshell2.png} 
				\caption{Postavljanje baze podataka} 
			\end{figure}
	
			\subsection*{Postavljanje i pokretanje backenda aplikacije}
			Backend aplikacije je pisan u Javi 17, što znači da je na sustavu koji pokreće aplikaciju potrebno imati instaliran JDK (Java Development Kit) - preporučujemo \href{https://www.openjdk.org}{\textit{OpenJDK}} kao open-source implementaciju.
			Osim Jave, potrebno je imati i instaliran \href{https://maven.apache.org/download.cgi}{\textit{Apache Maven}}. To je alat koji koristimo za upravljanje bibliotekama i "buildovima" backend aplikacije.
			
			Pri pokretanju, potrebno je u konzoli otvoriti mapu \textit{IzvorniKod/ozdravi-backend}, a aplikacija se dalje kompajlira izvršavanjem sljedećih naredbi:
			\lstinputlisting[style=ShellStyle, caption={Instalacija i kompajliranje backenda}]{code/mvn_cmd.shell}

			te se pokreće izvršavanjem sljedeće naredbe: 
			\lstinputlisting[style=ShellStyle, caption={Pokretanje backenda}]{code/java_jar.shell}
			
			Aplikacija je sada dostupna na portu 8080 lokalnog stroja.

			\subsection*{Postavljanje frontenda}
			Frontend aplikacije pisan je u Reactu te za njegovo pakiranje i pokretanje je potrebno preuzeti \href{https://nodejs.org/en/download/}{Node.js}.

			Pri pokretanju, potrebno je u konzoli otvoriti mapu \textit{IzvorniKod/ozdravi-frontend} i izvršiti sljedeće naredbe:
			\lstinputlisting[style=ShellStyle, caption={Instalacija i pokretanje frontenda}]{code/npm_cmd.shell}

			Prvu je naredbu potrebno izvršiti samo jednom, kako bi se prikupili potrebni paketi, a druga je naredba pokreće server na kojem je frontend dostupan za preuzimanje.
			
			\newpage \noindent Nakon pokretanja, aplikaciju je moguće vidjeti na adresi: \url{http://localhost:3000.}\\
			\begin{figure}[H]
				\includegraphics[width=\textwidth]{slike/loc3000.png} 
				\caption{Aplikacija na portu 3000} 
			\end{figure}
			Nakon svake promjene i spremanja koda, stranica se automatski ponovno učitava. Osim toga, postoji mogućnost promjene porta aplikacije uređivanjem datoteke \textit{package.json}, 
			gdje se u liniji koda \textit{"start": set PORT=3006 \&\& react-scripts start} može izmijeniti broj porta prema želji.
			
			
			\subsubsection*{Stvaranje PDF izvještaja dokumentacije}
			Dokumentacija za ovaj projekt je napisana koristeći paket \LaTeX  i nalazi se u mapi \textit{Dokumentacija/}. Kako bi se iz .tex datoteka izgradio PDF izvještaj potrebno je na računalu imati instalaciju \LaTeX-a i pokrenuti ispravne naredbe za izgradnju PDF-a (npr. \textit{latexmk}).
			\eject 
	\chapter{Zaključak i budući rad}
		
		%dio 2. revizije
		
		 %U ovom poglavlju potrebno je napisati osvrt na vrijeme izrade projektnog zadatka, koji su tehnički izazovi prepoznati, jesu li riješeni ili kako bi mogli biti riješeni, koja su znanja stečena pri izradi projekta, koja bi znanja bila posebno potrebna za brže i kvalitetnije ostvarenje projekta i koje bi bile perspektive za nastavak rada u projektnoj grupi.
		
		 %Potrebno je točno popisati funkcionalnosti koje nisu implementirane u ostvarenoj aplikaciji.

		 Zadatak grupe bio je razvoj web-aplikacije Ozdravi. Projekt "Ozdravi" predstavlja značajan korak prema modernizaciji zdravstvenog sustava, omogućujući brži pristup informacijama, efikasnije upravljanje resursima te bolje praćenje zdravstvenih podataka. Tijekom 13 tjedana rada, tim je uspješno implementirao zadanu web aplikaciju. Provedba projekta bila je podijeljena u tri faze. \\
		 Prva faza, poznata kao faza inicijacije \textit{(en. initiation)}, predstavlja početak cijelog procesa razvoja. Provedeno je formiranje tima, nakon čega je uslijedilo detaljno upoznavanje sa zadatkom. Time su članovi tima stekli jasnu sliku o svrsi aplikacije, funkcionalnostima i korisničkim zahtjevima. U ovoj fazi odabrane su odgovarajuće tehnologije i alati.   \\
		 Druga faza projekta, odnosno faza otkrivanja \textit{(engl. discovery)}, obilježena je strukturiranim pristupom definiranja zahtjeva i dokumentacijom, što je značajno olakšalo implementaciju u drugoj fazi. Prvi samostalni zadaci članova tima uključivali su usvajanje znanja o odabranim tehnologijama i alatima. Izrada vizualnih prikaza idejnih rješenja problemskog zadatka pomogla je u rješavanju nedoumica oko implementacije rješenja. Obrasci uporabe, sekvencijski dijagrami te modeli baze podataka bili su ključni u usmjeravanju podtimova zaduženih za razvoj backenda i frontenda. \\
		 U trećoj fazi, poznata kao faza isporuke \textit{(engl. delivery)}, naglasak je bio na samostalnom radu članova tima na implementaciji i dokumentaciji. Kvalitetna dokumentacija olakšat će buduće prilagodbe sustava te omogućiti korisnicima lakše korištenje aplikacije. Osim realizacije rješenja, u ovoj fazi bilo je potrebno izraditi ostale UML dijagrame te popratnu dokumentaciju kako bi se budućim korisnicima omogućilo jednostavnije korištenje te izvršenje potrebnih preinaka sustava. 
		 Redoviti sastanci tima pridonijeli su informiranosti o napretku projekta, izvrsnoj organizaciji te učinkovitoj podjeli zadataka. \\
		 Nakon uspješne implementacije, projekt "Ozdravi" otvara prostor za daljnje poboljšanje i proširenje. U skladu s time, moguće je izvesti refaktorizaciju određenih dijelova koda kako bi se smanjio tehnički dug i poboljšala održivost aplikacije. Također, razmatra se unapređenje deployment okruženja kako bi se postigle bolje performanse i povećala dostupnost sustava. Nužno je i uvesti CI/CD (Continous Integration/Continous Delivery) sustava koji bi mogao automatski napraviti deploy aplikacije na javni poslužitelj u slučaju commita u granu "main" i pritom osigurati kvalitetu isporučenog koda izvršavanjem testova.
		 Potencijalno proširenje funkcionalnosti sustava obuhvaća implementaciju mobilne aplikacije, pružajući dodatnu fleksibilnost korisnicima. Nadogradnja dodatnim funkcionalnostima ključna je u zadovoljavanju potreba korisnika, omogućavajući dodavanje novih značajki koje će obogatiti korisničko iskustvo. Uz to, postoji mogućnost unapređenja korisničkog sučelja (UI) i korisničkog iskustva (UX), čime će se poboljšati funkcionalnost sustava. S kontinuiranim praćenjem povratnih informacija korisnika, osigurat će se da aplikacija "Ozdravi" uvijek bude u skladu s najvišim standardima i potrebama moderne zdravstvene tehnologije. \\
		 Sudjelovanje u projektu "Ozdravi" pružilo je vrijedno iskustvo članovima tima, naglašavajući važnost suradnje, organizacije te kontinuiranog učenja u radu na složenim projektima.
		\eject 
	\include{Literatura}
	
	\begingroup
	\renewcommand*\listfigurename{Indeks slika i dijagrama}
	%\renewcommand*\listtablename{Indeks tablica}
	%\let\clearpage\relax
	\listoffigures
	%\vspace{10mm}
	%\listoftables
	\endgroup
	\addcontentsline{toc}{chapter}{Indeks slika i dijagrama}

	\eject 
	\chapter*{Dodatak: Prikaz aktivnosti grupe}
		\addcontentsline{toc}{chapter}{Dodatak: Prikaz aktivnosti grupe}
		
		\section*{Dnevnik sastajanja}
		
		\begin{packed_enum}
			\item  sastanak
			
			\item[] \begin{packed_item}
				\item Datum: \date[{19. listopada 2023.} %%zas ovdje ide ova sintaksa [
				\item Prisustvovali: D. Matić, L. Crvelin, V. Kumanović, M. Lešković, K. Valečić, M. Vidaković
				\item Teme sastanka:
				\begin{packed_item}
					\item  izbor tehnologije
					\item  izbor teme
                    \item  ime tima
                    \item  osnovna podjela posla
                    \item  osiguravanje pristupa na sve servisa svim članovima tima
				\end{packed_item}
            \item Zaključak: \\Dogovoreni su taskovi za prvi sprint (do četvrtrka 26.10). Oni za većinu ljudi uključuju učenje gita i tehnologija koje su odabrali (backend/frontend), a L. Crvelin treba i napraviti prve korake u organizaciji dokumentacije. Voditelj D. Matić će postaviti repozitorij na GitHubu i osigurati svima pritup. Dogovorena je i radionica na temu značajki (featurea) aplikacije sutra u 12 sati, gdje ćemo pokušati prioritizirati posao i složiti Jiru.
			\end{packed_item}

   
			\item  sastanak
			\item[] \begin{packed_item}
				\item Datum: \date[{20. listopada 2023.}
				\item Prisustvovali:  D. Matić, L. Crvelin, V. Kumanović, M. Lešković, K. Valečić, M. Vidaković
				\item Teme sastanka:
				\begin{packed_item}
					\item  analiziranje značajki aplikacije i napisati iz njih epice i storije u Jiri
					\item  stvaranje dijagrama
                    \item  daljnja podjela posla
				\end{packed_item}
            \item Zaključak: \\Detaljno su analizirane značajke aplikacije i stvoreni dijagrami iz kojih se može stvoriti definicija baze podataka što će do kraja sprinta učiniti V. Kumanović i prioritizirani popis zadataka koji će u Jiri stvoriti D. Matić. Također, identificirali smo nejasnoće u samoj definiciji zadatka koje ćemo prenijeti asistentici.
            M. Vidaković će do kraja sprinta istražiti kako se pravilno postavlja Spring projekt, a K. Valečić će isto učiniti za React u čemu će mu pomoći M. Lešković. \\
            Dogovorene su još dvije radionice sljedeći tjedan, u četvrtak i petak na temu postavljanja Reacta i Springa.
			\end{packed_item}


            \item  sastanak
			\item[] \begin{packed_item}
				\item Datum: \date[{26. listopada 2023.}
				\item Prisustvovali: D. Matić, L. Crvelin, L. Cvetkovski, V. Kumanović, M. Lešković, K. Valečić, M. Vidaković 
				\item Teme sastanka:
				\begin{packed_item}
					\item  planiranje i praćenje napretka
					\item  postavljanje backend-a prema svim pravilima i omogućavanje nesmetanog razvoja u budućnosti.
				\end{packed_item}
            \item Zaključak: \\
            Potvrdili smo da svi imaju potreban pristup na JIRU i GitHub repozitoriju. \\
             L. Crvelin je izložila napredak u dokumentiranju i predstavila način prikupljanja podataka za pojedine dijelove dokumentacije. Podaci o napretku će se prikupljati kroz komentare na JIRA taskovima i kroz bilješke sa sastanaka. Sljedeći koraci vezani uz izradu dokumentacije su raspisivanje obrazaca uporabe i pisanje opisa projektnog zadatka. To će preuzeti L. Crvelin i V. Kumanović. Također, L. Crvelin je naglasila važnost praćenja potrošenog vremena za svaki zadatak koji radimo. To će u JIRI postaviti D. Matić. \\
             Predstavljena je i schema baze podataka koju smo komentirali i utvrdili koje su potrebne promjene. Te će promjene u ovom sprintu učiniti V. Kumanović i M. Vidaković te će ih predstaviti timu na sljedećem planiranju sprinta u četvrtak. \\
             M. Vidaković je prezentirala napredak vezan uz postavljanje Spring backenda i preuzela zadatak integracije backenda s Postgresom što se može događati usporedno uz razvijanje scheme baze. \\
             K. Valečić i M. Lešković su potvrdili da su spremni za nadolazeću radionicu vezanu uz postavljanje Reacta, koja će se održati sutra (petak, 27.10. u 14 sati). Nakon te radionice bit ćemo spremni za početak razvoja na frontendu, što će preuzeti L. Cvetkovski i K. Valečić. \\
             Komentirali smo rad s gitom te je M. Lešković pokazao workflow u GitHub Desktop alatu. Naglašena je važnost da više ljudi pogleda PR prije mergea s glavnom granom.
            
			\end{packed_item}

               \item  sastanak
			\item[] \begin{packed_item}
				\item Datum: \date[{27. listopada 2023.}
				\item Prisustvovali:  
				\item Teme sastanka:
				\begin{packed_item}
					\item  planiranje i praćenje napretka
					\item  postavljanje frontend-a prema svim pravilima i omogućavanje nesmetanog razvoja u budućnosti
				\end{packed_item}
            \item Zaključak
			\end{packed_item}
			
		\end{packed_enum}
		
		\eject
		\section*{Tablica aktivnosti}
			 %Doprinose u aktivnostima treba navesti u satima po članovima grupe po aktivnosti.}

			\begin{longtblr}[
					label=none,
				]{
					vlines,hlines,
					width = \textwidth,
					colspec={X[7, l]X[1, c]X[1, c]X[1, c]X[1, c]X[1, c]X[1, c]X[1, c]}, 
					vline{1} = {1}{text=\clap{}},
					hline{1} = {1}{text=\clap{}},
					rowhead = 1,
				} 
			
				\SetCell[c=1]{c}{} & \SetCell[c=1]{c}{\rotatebox{90}{\textbf{Dorian Matić}}} & \SetCell[c=1]{c}{\rotatebox{90}{\textbf{Lucia Crvelin }}} &	\SetCell[c=1]{c}{\rotatebox{90}{\textbf{Leon Cvetkovski }}} & \SetCell[c=1]{c}{\rotatebox{90}{\textbf{Vedran Kumanović }}} &	\SetCell[c=1]{c}{\rotatebox{90}{\textbf{Mihael Lešković }}} & \SetCell[c=1]{c}{\rotatebox{90}{\textbf{Krešimir Valečić }}} &	\SetCell[c=1]{c}{\rotatebox{90}{\textbf{Mia Vidaković }}} \\  
				Upravljanje projektom 		&  &  &  &  &  &  & \\ 
				Opis projektnog zadatka 	&  & 3 &  &  &  &  & \\ 
				
				Funkcionalni zahtjevi       &  &  &  &  &  &  &  \\ 
				Opis pojedinih obrazaca 	&  &  &  &  &  &  &  \\ 
				Dijagram obrazaca 			&  &  &  &  &  &  &  \\ 
				Sekvencijski dijagrami 		&  &  &  &  &  &  &  \\ 
				Opis ostalih zahtjeva 		&  &  &  &  &  &  &  \\ 

				Arhitektura i dizajn sustava	 &  &  &  &  &  &  &  \\ 
				Baza podataka				&  &  &  &  &  &  &   \\ 
				Dijagram razreda 			&  &  &  &  &  &  &   \\ 
				Dijagram stanja				&  &  &  &  &  &  &  \\ 
				Dijagram aktivnosti 		&  &  &  &  &  &  &  \\ 
				Dijagram komponenti			&  &  &  &  &  &  &  \\ 
				Korištene tehnologije i alati 		&  &  &  &  &  &  &  \\ 
				Ispitivanje programskog rješenja 	&  &  &  &  &  &  &  \\ 
				Dijagram razmještaja			&  &  &  &  &  &  &  \\ 
				Upute za puštanje u pogon 		&  &  &  &  &  &  &  \\  
				Dnevnik sastajanja 			&  &  &  &  &  &  &  \\ 
				Zaključak i budući rad 		&  &  &  &  &  &  &  \\  
				Popis literature 			&  &  &  &  &  &  &  \\  
				&  &  &  &  &  &  &  \\ \hline 
				\textit{Dodatne stavke kako ste podijelili izradu aplikacije} 			&  &  &  &  &  &  &  \\ 
				\textit{npr. izrada početne stranice} 				&  &  &  &  &  &  &  \\  
				\textit{izrada baze podataka} 		 			&  &  &  &  &  &  & \\  
				\textit{spajanje s bazom podataka} 							&  &  &  &  &  &  &  \\ 
				\textit{back end} 							&  &  &  &  &  &  &  \\  
				 							&  &  &  &  &  &  &\\ 
			\end{longtblr}
					
					
		\eject
		\section*{Dijagrami pregleda promjena}
		
		%dio 2. revizije
		
		%Prenijeti dijagram pregleda promjena nad datotekama projekta. Potrebno je na kraju projekta generirane grafove s gitlaba prenijeti u ovo poglavlje dokumentacije. Dijagrami za vlastiti projekt se mogu preuzeti s gitlab.com stranice, u izborniku Repository, pritiskom na stavku Contributors.
		
	
\end{document} %naredbe i tekst nakon ove naredbe ne ulaze u izgrađen dokument 

