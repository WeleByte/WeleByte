\chapter{Zaključak i budući rad}
		
		%dio 2. revizije
		
		 %U ovom poglavlju potrebno je napisati osvrt na vrijeme izrade projektnog zadatka, koji su tehnički izazovi prepoznati, jesu li riješeni ili kako bi mogli biti riješeni, koja su znanja stečena pri izradi projekta, koja bi znanja bila posebno potrebna za brže i kvalitetnije ostvarenje projekta i koje bi bile perspektive za nastavak rada u projektnoj grupi.
		
		 %Potrebno je točno popisati funkcionalnosti koje nisu implementirane u ostvarenoj aplikaciji.

		 Zadatak grupe bio je razvoj web-aplikacije Ozdravi. Projekt "Ozdravi" predstavlja značajan korak prema modernizaciji zdravstvenog sustava, omogućujući brži pristup informacijama, efikasnije upravljanje resursima te bolje praćenje zdravstvenih podataka. Tijekom 13 tjedana rada, tim je uspješno implementirao zadanu web aplikaciju. Provedba projekta bila je podijeljena u tri faze. \\
		 Prva faza, poznata kao faza inicijacije \textit{(en. initiation)}, predstavlja početak cijelog procesa razvoja. Provedeno je formiranje tima, nakon čega je uslijedilo detaljno upoznavanje sa zadatkom. Time su članovi tima stekli jasnu sliku o svrsi aplikacije, funkcionalnostima i korisničkim zahtjevima. U ovoj fazi odabrane su odgovarajuće tehnologije i alati.   
		 Druga faza projekta, odnosno faza otkrivanja \textit{(engl. discovery)}, obilježena je strukturiranim pristupom definiranja zahtjeva i dokumentacijom, što je značajno olakšalo implementaciju u drugoj fazi. Prvi samostalni zadaci članova tima uključivali su usvajanje znanja o odabranim tehnologijama i alatima. Izrada vizualnih prikaza idejnih rješenja problemskog zadatka pomogla je u rješavanju nedoumica oko implementacije rješenja. Obrasci uporabe, sekvencijski dijagrami te modeli baze podataka bili su ključni u usmjeravanju podtimova zaduženih za razvoj backenda i frontenda. \\
		 U trećoj fazi, poznata kao faza isporuke \textit{(engl. delivery)}, naglasak je bio na samostalnom radu članova tima na implementaciji i dokumentaciji. Kvalitetna dokumentacija olakšat će buduće prilagodbe sustava te omogućiti korisnicima lakše korištenje aplikacije. Osim realizacije rješenja, u ovoj fazi bilo je potrebno izraditi ostale UML dijagrame te popratnu dokumentaciju kako bi se budućim korisnicima omogućilo jednostavnije korištenje te izvršenje potrebnih preinaka sustava. 
		 Redoviti sastanci tima pridonijeli su informiranosti o napretku projekta, izvrsnoj organizaciji te učinkovitoj podjeli zadataka. \\
		 Nakon uspješne implementacije, projekt "Ozdravi" otvara prostor za daljnje poboljšanje i proširenje. U skladu s time, moguće je izvesti refaktorizaciju određenih dijelova koda kako bi se smanjio tehnički dug i poboljšala održivost aplikacije. Također, razmatra se unapređenje deployment okruženja kako bi se postigle bolje performanse i povećala dostupnost sustava.
		 Potencijalno proširenje funkcionalnosti sustava obuhvaća implementaciju mobilne aplikacije, pružajući dodatnu fleksibilnost korisnicima. Nadogradnja dodatnim funkcionalnostima ključna je u zadovoljavanju potreba korisnika, omogućavajući dodavanje novih značajki koje će obogatiti korisničko iskustvo. Uz to, postoji mogućnost unapređenja korisničkog sučelja (UI) i korisničkog iskustva (UX), čime će se poboljšati funkcionalnost sustava. S kontinuiranim praćenjem povratnih informacija korisnika, osigurat će se da aplikacija "Ozdravi" uvijek bude u skladu s najvišim standardima i potrebama moderne zdravstvene tehnologije. \\
		 Sudjelovanje u projektu "Ozdravi" pružilo je vrijedno iskustvo članovima tima, naglašavajući važnost suradnje, organizacije te kontinuiranog učenja u radu na složenim projektima.
		\eject 