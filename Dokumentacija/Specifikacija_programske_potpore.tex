\chapter{Specifikacija programske potpore}
		
	\section{Funkcionalni zahtjevi}
			
			\noindent \textbf{Dionici:}
			
			\begin{packed_enum}
				
				\item Roditelj
				\item Pedijatar
                \item Liječnik obiteljske medicine
                \item Administrator
                
			\end{packed_enum}
			
			\noindent \textbf{Aktori i njihovi funkcionalni zahtjevi:}
			
			
			\begin{packed_enum}
				\item  \underbar{Roditelj (inicijator) može:}
				
				\begin{packed_enum}
    
					\item registrirati se i prijaviti u sustav
                    \item vidjeti profil za sebe i svoju djecu
					\item pregledati medicinske nalaze, dijagnoze i povijesti posjeta za svoju djecu
                    \item zatražiti povratnu informaciji liječnika ili pedijatra
                    \item pregledati svoje povijesti posjeta kod liječnika i dijagnoze
                    \item vidjeti druga mišljenja koja je stvorio
                    
				\end{packed_enum}
			
				\item  \underbar{Pedijatar (inicijator) može:}
				
				\begin{packed_enum}
			
					\item registrirati se i prijaviti u sustav kako bi upisao podatke za djecu
                    \item prijaviti djecu po identifikatoru (OIB)
                    \item pregledati popis sve prijavljene djece i njihove kartone
					\item unijeti medicinske podatke (dijagnoze, preglede, terapije) i ažurirati te podatke
                    \item evidentirati preporuku za bolovanje za roditelja u slučaju bolesti djeteta
                    \item utvrditi bolest djeteta nakon čega se šalje ispričnica u školu ili vrtić
                    \item naručiti djecu na specijalističke preglede/postupke
                    \item dati očitovanje na drugo mišljenje
                
				\end{packed_enum}

                \item  \underbar{Liječnik obiteljske medicine (inicijator) može:}
				
				\begin{packed_enum}
			
					\item registrirati se i prijaviti u sustav
                    \item pregledati i odobriti doznake izdane od strane pedijatra koje se šalju poslodavcu roditelja
                    \item pristupiti povijesti medicinskih podataka
                    \item dati očitovanje na drugo mišljenje
                    \item naručiti pacijenta na specijalistički pregled/postupak
                    \item unositi podatke o pregledima
                    \item pristupiti medicinskim nalazima i upitima 
                    
				\end{packed_enum}
    
                \item  \underbar{Administrator (inicijator) može:}
				
				\begin{packed_enum}

                    \item registracija i prijava u sustav kao administrator
					\item pripremiti registre djece (osnovni podaci, OIB)
                    \item po registraciji roditelja, povezati iste s djecom (preko OIB-a)
					\item administrirati korisničke račune
                    \item ažurirati sve podatke u aplikaciji
                 
				\end{packed_enum}

                \item  \underbar{Baza podataka (sudionik):}
				
				\begin{packed_enum}

                    \item pohranjuje sve podatke o korisnicima i njihovim ovlastima
					
				\end{packed_enum}
    
			\end{packed_enum}
			
			\eject 
			
			
				
			\subsection{Obrasci uporabe}
				
				%dio 1. revizije
				
				\subsubsection{Opis obrazaca uporabe}
					%Funkcionalne zahtjeve razraditi u obliku obrazaca uporabe. Ukoliko u nekom koraku može doći do odstupanja, potrebno je to odstupanje opisati i po mogućnosti ponuditi rješenje kojim bi se tijek obrasca vratio na osnovni tijek.
					

					\noindent \underbar{\textbf{UC$<$broj obrasca$>$ -$<$ime obrasca$>$}}
					\begin{packed_item}
	
						\item \textbf{Glavni sudionik: }$<$sudionik$>$
						\item  \textbf{Cilj:} $<$cilj$>$
						\item  \textbf{Sudionici:} $<$sudionici$>$
						\item  \textbf{Preduvjet:} $<$preduvjet$>$
						\item  \textbf{Opis osnovnog tijeka:}
						
						\item[] \begin{packed_enum}
	
							\item $<$opis korak jedan$>$
							\item $<$opis korak dva$>$
							\item $<$opis korak tri$>$
							\item $<$opis korak četiri$>$
							\item $<$opis korak pet$>$
						\end{packed_enum}
						
						\item  \textbf{Opis mogućih odstupanja:}
						
						\item[] \begin{packed_item}
	
							\item[2.a] $<$opis mogućeg scenarija odstupanja u koraku 2$>$
							\item[] \begin{packed_enum}
								
								\item $<$opis rješenja mogućeg scenarija korak 1$>$
								\item $<$opis rješenja mogućeg scenarija korak 2$>$
								
							\end{packed_enum}
							\item[2.b] $<$opis mogućeg scenarija odstupanja u koraku 2$>$
							\item[3.a] $<$opis mogućeg scenarija odstupanja  u koraku 3$>$
							
						\end{packed_item}
					\end{packed_item}
				
					
				\subsubsection{Dijagrami obrazaca uporabe}
					
					%Prikazati odnos aktora i obrazaca uporabe odgovarajućim UML dijagramom. Nije nužno nacrtati sve na jednom dijagramu. Modelirati po razinama apstrakcije i skupovima srodnih funkcionalnosti.
				\eject		
				
			\subsection{Sekvencijski dijagrami}
				
				%dio 1. revizije
				
				%Nacrtati sekvencijske dijagrame koji modeliraju najvažnije dijelove sustava (max. 4 dijagrama). Uz svaki dijagram napisati detaljni opis dijagrama.
    
				\eject
	
		\section{Ostali zahtjevi}
		
			%dio 1. revizije
		 
			 %Nefunkcionalni zahtjevi i zahtjevi domene primjene dopunjuju funkcionalne zahtjeve. Oni opisuju kako se sustav treba ponašati i koja ograničenja treba poštivati (performanse, korisničko iskustvo, pouzdanost, standardi kvalitete, sigurnost...). Primjeri takvih zahtjeva u Vašem projektu mogu biti: podržani jezici korisničkog sučelja, vrijeme odziva, najveći mogući podržani broj korisnika, podržane web/mobilne platforme, razina zaštite (protokoli komunikacije, kriptiranje...)... Svaki takav zahtjev potrebno je navesti u jednoj ili dvije rečenice.
			 
			 
			 
	