\chapter{Specifikacija programske potpore}
	\section{Funkcionalni zahtjevi}
			\noindent \textbf{Dionici:}
			\begin{packed_enum}
				\item Roditelj
				\item Pedijatar
                \item Liječnik obiteljske medicine
                \item Administrator
			\end{packed_enum}
			
			\noindent \textbf{Aktori i njihovi funkcionalni zahtjevi:}
			\begin{packed_enum}
				\item  \underbar{Roditelj (inicijator) može:}
				\begin{packed_enum}
					\item registrirati se i prijaviti u sustav
                    \item vidjeti profil za sebe i svoju djecu
					\item pregledati medicinske nalaze, dijagnoze i povijesti pregleda za sebe i svoju djecu
                    \item zatražiti povratnu informaciju liječnika ili pedijatra o nalazu dobivenim temeljem određene usluge u privatnoj ustanovi, tj. drugog mišljenja
                    \item vidjeti druga mišljenja koja je stvorio
				\end{packed_enum}
			
				\item  \underbar{Pedijatar (inicijator) može:}
				\begin{packed_enum}
					\item prijaviti se u sustav
                    \item prijaviti djecu po identifikatoru (OIB)
                    \item pregledati popis sve prijavljene djece i njihove kartone
					\item unijeti medicinske podatke (dijagnoze, preglede, terapije) i ažurirati te podatke
                    \item evidentirati preporuku za bolovanje za roditelja u slučaju bolesti djeteta
                    \item utvrditi bolest djeteta nakon čega se šalje ispričnica u vrtić
                    \item naručiti djecu na specijalističke preglede/postupke
                    \item dati povratnu informaciju na drugo mišljenje
				\end{packed_enum}

                \item  \underbar{Liječnik obiteljske medicine (inicijator) može:}
				\begin{packed_enum}
					\item prijaviti se u sustav
                    \item pregledati i odobriti doznake izdane od strane pedijatra koje se šalju poslodavcu roditelja
                    \item utvrditi bolest djeteta nakon čega se šalje ispričnica u školu
                    \item dati povratnu informaciju na drugo mišljenje
                    \item naručiti pacijenta na specijalistički pregled/postupak
                    \item unositi podatke o pregledima
                    \item pristupiti medicinskim nalazima, povijesti pregleda i upitima 
				\end{packed_enum}
    
                \item  \underbar{Administrator (inicijator) može:}
				\begin{packed_enum}
                    \item prijava u sustav kao administrator
					\item pripremiti registre djece (osnovni podaci, OIB)
                    \item registrira liječnika obiteljske medicine i pedijatra
                    \item po registraciji roditelja, povezati iste s djecom (preko OIB-a)
					\item administrirati korisničke račune
                    \item ažurirati sve podatke u aplikaciji
				\end{packed_enum}

                \item  \underbar{Baza podataka (sudionik):}
				\begin{packed_enum}
                    \item pohranjuje sve podatke o korisnicima i njihovim ovlastima
				\end{packed_enum}
			\end{packed_enum}
			
			\eject 
				
			\subsection{Obrasci uporabe}
				
				%dio 1. revizije
				
				\subsubsection{Opis obrazaca uporabe}
					%Funkcionalne zahtjeve razraditi u obliku obrazaca uporabe. Ukoliko u nekom koraku može doći do odstupanja, potrebno je to odstupanje opisati i po mogućnosti ponuditi rješenje kojim bi se tijek obrasca vratio na osnovni tijek.
					

					\noindent \underbar{\textbf{UC1 - registraj se}}
					\begin{packed_item}
	
						\item \textbf{Glavni sudionik: }roditelj
						\item  \textbf{Cilj:} stvoriti korisnički račun za pristup sustavu
						\item  \textbf{Sudionici:} baza podataka
						\item  \textbf{Preduvjet:} -
						\item  \textbf{Opis osnovnog tijeka:}
						
						\item[] \begin{packed_enum}
	
							\item korisnik odabire opciju za registraciju
							\item korisnik unosi potrebne korisničke podatke
							\item korisnik dobiva prstup korisničkim funkcijama

						\end{packed_enum}
						
						\item  \textbf{Opis mogućih odstupanja:}
						
						\item[] \begin{packed_item}
	
							\item[2.a] Odabir već zauzetog korisničkog imena i/ili e-maila, unos korisničkih podatka u nedozvoljenom formatu
							\item[] \begin{packed_enum}
								
								\item sustav obavještava korisnika o neuspjelom upisu i vraća ga na stranicu za registraciju
								\item korisnik mijenja potrebne podatke te završava unos ili odustaje od registracije
							\end{packed_enum}
							
						\end{packed_item}
					\end{packed_item}

                    \noindent \underbar{\textbf{UC2 - prijavi se}}
					\begin{packed_item}
	
						\item \textbf{Glavni sudionik: }korisnik
						\item  \textbf{Cilj: }dobiti pristup korisničkom sučelju
						\item  \textbf{Sudionici: }baza podataka
						\item  \textbf{Preduvjet: }registracija
						\item  \textbf{Opis osnovnog tijeka:}
						
						\item[] \begin{packed_enum}
	
							\item unos korisničkog imena i lozinke
							\item potvrda o ispravnosti unesenih podataka
							\item pristup korisničkim funkcijama

						\end{packed_enum}
						
						\item  \textbf{Opis mogućih odstupanja:}
						
						\item[] \begin{packed_item}
	
							\item[2.a] Neispravno korisničko ime/lozinka ili ne postoji profil s ovim korisničkim imenom/e-mailom
							\item[] \begin{packed_enum}
								
								\item sustav obavještava korisnika o neuspjelom upisu i vraća ga na stranicu za prijavu
								
							\end{packed_enum}
							
						\end{packed_item}
					\end{packed_item}

                    \noindent \underbar{\textbf{UC3 - pregledaj profila djeteta}}
					\begin{packed_item}
	
						\item \textbf{Glavni sudionik: }roditelj
						\item  \textbf{Cilj:} pregledati djetetovu povijest pregleda, nalaze i dijagnoze
						\item  \textbf{Sudionici:} baza podataka
						\item  \textbf{Preduvjet:} roditelj je prijavljen, ima dijete i povezani su u sustavu
						\item  \textbf{Opis osnovnog tijeka:}
						
						\item[] \begin{packed_enum}
	
							\item roditelj odabire opciju 'moja djeca' te mu se prikaže popis njegove djece
							\item roditelj odabire dijete čije podatke želi pregledati
							
						\end{packed_enum}
						
					\end{packed_item}

                    
                    \noindent \underbar{\textbf{UC4 - pregledaj vlastiti profil}}
					\begin{packed_item}
	
						\item \textbf{Glavni sudionik: }roditelj, pedijatar, liječnik obiteljske medicine, administrator
						\item  \textbf{Cilj:} pregledati vlastiti profil
						\item  \textbf{Sudionici:} baza podataka
						\item  \textbf{Preduvjet:} korisnik je prijavljen 
						\item  \textbf{Opis osnovnog tijeka:}
						
						\item[] \begin{packed_enum}
	
							\item korisnik odabire opciju 'osobni podaci'
							\item korisnik vidi svoje podatke
							
						\end{packed_enum}
						
					\end{packed_item}


                     \noindent \underbar{\textbf{UC5 - pregledaj medicinske podatake}}
					\begin{packed_item}
	
						\item \textbf{Glavni sudionik: }roditelj
						\item  \textbf{Cilj:} pristupiti povijesti pregleda, za svaki pregled su vidljivi nalazi i dijagnoze ustanovljene na pregledu
						\item  \textbf{Sudionici:} baza podataka
						\item  \textbf{Preduvjet:} roditelj je prijavljen u sustav
						\item  \textbf{Opis osnovnog tijeka:}
						
						\item[] \begin{packed_enum}
	
							\item roditelj odabire opciju 'moji pregledi'
							\item roditelj vidi listu svojih prošlih i zakazanih pregleda
							\item odabirom na pojedini pregled, roditelj vidi nalaze i dijagnoze s tog pregleda

						\end{packed_enum}
					\end{packed_item}

                    \noindent \underbar{\textbf{UC6 - stvori zahtjev za drugim mišljenjem}}
					\begin{packed_item}
						\item \textbf{Glavni sudionik: }roditelj
						\item  \textbf{Cilj:} zatražiti drugo mišljenja liječnika ili pedijatra o nalazu dobivenom temeljem određene usluge u privatnoj ustanovi
						\item  \textbf{Sudionici:} baza podataka
						\item  \textbf{Preduvjet:} roditelj je prijavljen u sustav
						\item  \textbf{Opis osnovnog tijeka:}
						
						\item[] \begin{packed_enum}
	
							\item roditelj unosi nalaz dobiven u privatnoj ustanovi
                            \item roditelj odabire liječnika/peidjatra čije drugo mišljenje želi dobiti
							\item roditelj šalje zahtjev za drugim mišljenjem
						
						\end{packed_enum}
						
						\item  \textbf{Opis mogućih odstupanja:}
						
						\item[] \begin{packed_item}
	
							\item[2.a] pacijent (roditelj/dijete) ne pripada nijednom liječniku
							\item[] \begin{packed_enum}
								\item roditelj nije u mogućnosti poslati zahtjev, sve dok pacijent pripada nijednom liječniku
							\end{packed_enum}

                            \item[3.a] neispravno popunjen zahtjev
                            \item[] \begin{packed_enum}
								\item zahtjev nije moguće poslati te se roditelja ponovno usmjerava na ispunjavanje zahtjeva za drugim mišljenjem
							\end{packed_enum}
						\end{packed_item}
					\end{packed_item}


                    \noindent \underbar{\textbf{UC7 - Pregledaj zahtjev za drugim mišljenjem}}
					\begin{packed_item}
	
						\item \textbf{Glavni sudionik: }roditelj
						\item  \textbf{Cilj:} pregledati predani zahtjev za drugim mišljenjem te liječnikovu ili pedijatrovu povratnu informaciju 
						\item  \textbf{Sudionici:} baza podataka
						\item  \textbf{Preduvjet:} roditelj je prijavljen u sustav, predao je zahtjev za drugim mišljenjem
						\item  \textbf{Opis osnovnog tijeka:}
						
						\item[] \begin{packed_enum}
	
							\item roditelj odabire opciju 'Moji zahtjevi'
							\item pregledava popis svojih zahtjeva
							\item odabire zahtjev koji želi vidjeti
                            \item pregledava liječnikovo/pedijatrovo drugo mišljenje

						\end{packed_enum}
						
						\item  \textbf{Opis mogućih odstupanja:}
						
						\item[] \begin{packed_item}
	
							\item[3.a] liječnik/pedijatar nije poslao svoje drugo mišljenje
							\item[] \begin{packed_enum}
								\item roditelj može vidjeti samo svoj predani zahtjev
							\end{packed_enum}
							
						\end{packed_item}
					\end{packed_item}

                    \noindent \underbar{\textbf{UC8 - prijavi dijete po identifikatoru }}
					\begin{packed_item}
	
						\item \textbf{Glavni sudionik: }pedijatar
						\item  \textbf{Cilj:} prijaviti dijete u sustav
						\item  \textbf{Sudionici:} baza podataka
						\item  \textbf{Preduvjet:} pedijatar je prijavljen u sustav
						\item  \textbf{Opis osnovnog tijeka:}
						
						\item[] \begin{packed_enum}
	
							\item pedijatar vidi popis djece koje nemaju pridijeljenog pedijatra
							\item pedijatar odabire dijete koje postaje njegov pacijent

						\end{packed_enum}
						
						\item  \textbf{Opis mogućih odstupanja:}
						
						\item[] \begin{packed_item}
	
							\item[1.a] ne postoji dijete koje nema pridijeljenog pedijatra 
							\item[] \begin{packed_enum}
								\item pedijatar ne može odabrati nijedno dijete
							\end{packed_enum}
							
						\end{packed_item}
					\end{packed_item}

     
                    \noindent \underbar{\textbf{UC9 - pregledaj prijavljene pacijenate i  njihove kartone }}
					\begin{packed_item}
	
						\item \textbf{Glavni sudionik: }pedijatar, liječnik obiteljske medicine
						\item  \textbf{Cilj:} pregledati popis prijavljenih pacijenata i njihovih kartona
						\item  \textbf{Sudionici:} baza podataka
						\item  \textbf{Preduvjet:} liječnik je prijavljen u sustav
						\item  \textbf{Opis osnovnog tijeka:}
						
						\item[] \begin{packed_enum}
	
							\item liječnik pregledava popis svojih pacijenata
							\item liječnik odabire pacijenta čiji karton želi pregledati
						\end{packed_enum}
						
						\item  \textbf{Opis mogućih odstupanja:}
						
						\item[] \begin{packed_item}
	
							\item[2.a] liječnik nema nijednog pacijenta
							\item[] \begin{packed_enum}
								
								\item ne može pregledati pacijentov karton, sve dok pacijent nije prijavljen u njegovoj ordinaciji
							\end{packed_enum}
							
						\end{packed_item}
					\end{packed_item}


                    \noindent \underbar{\textbf{UC10 - unesi pregled}}
					\begin{packed_item}
	
						\item \textbf{Glavni sudionik: }pedijatar, liječnik obiteljske medicine
						\item  \textbf{Cilj:} unijeti pregled te medicinske podatke vezane uz pregled: dijagnoza, nalaz, terapija
						\item  \textbf{Sudionici:} baza podataka
						\item  \textbf{Preduvjet:} liječnik je prijavljen u sustav, pacijent se nalazi na njegovom popisu
						\item  \textbf{Opis osnovnog tijeka:}
						
						\item[] \begin{packed_enum}
	
							\item liječnik pregledava popis svojih pacijenata
							\item liječnik odabire pacijenta za kojega želi unijeti novi pregled
							\item liječnik unosi podatke o pregledu

						\end{packed_enum}
					
					\end{packed_item}

                    \noindent \underbar{\textbf{UC11 - evidentiraj preporuku za bolovanje}}
					\begin{packed_item}
	
						\item \textbf{Glavni sudionik: }pedijatar
						\item  \textbf{Cilj:} evidentirati preporuku za bolovanje roditelja u slučaju bolesti djeteta
						\item  \textbf{Sudionici:} baza podataka
						\item  \textbf{Preduvjet:} pedijatar je prijavljen u sustav, stvoren je pregled na kojem je dijagnosticirana bolest djeteta
						\item  \textbf{Opis osnovnog tijeka:}
						
						\item[] \begin{packed_enum}
	
							\item pedijatar odabire opciju 'Nova preporuka'
							\item pedijatar odabire pregled temeljem kojeg je potrebna preporuka za bolovanje
							\item preporuka se šalje roditeljevom liječniku obiteljske medicine

						\end{packed_enum}
                    \end{packed_item}
						
                    \noindent \underbar{\textbf{UC12 - pošalji ispričnicu u vrtić/školu}}
					\begin{packed_item}
	
						\item \textbf{Glavni sudionik: }pedijatar
						\item  \textbf{Cilj:} poslati ispričnicu zbog bolesti djeteta u njegov vrtić/školu
						\item  \textbf{Sudionici:} baza podataka
						\item  \textbf{Preduvjet:} liječnik je prijavljen u sustav, stvoren je pregled na kojem je dijagnosticirana bolest djeteta
						\item  \textbf{Opis osnovnog tijeka:}
						
						\item[] \begin{packed_enum}
	
							\item pedijatar odabire opciju 'Nova ispričnica'
							\item pedijatar odabire pregled na temelju kojega je potrebno poslati ispričnicu
							\item unosi se e-mail adresa na koju je potrebno poslati ispričnicu
                            \item ispričnica je poslana

						\end{packed_enum}
						
						\item  \textbf{Opis mogućih odstupanja:}
						
						\item[] \begin{packed_item}
	
							\item[3.a] pedijatar je unio pogrešnu e-mail adresa
							\item[] \begin{packed_enum}
								
								\item pedijatra se vraća na ponovni unos e-mail adrese
							\end{packed_enum}
                            \item[3.b] pedijatar je neispravno popunio ispričnicu
							\item[] \begin{packed_enum}
								
								\item pedijatra se vraća na ponovno popunjavanje ispričnice
							\end{packed_enum}
							
						\end{packed_item}
					\end{packed_item}

     
                    \noindent \underbar{\textbf{UC13 - naruči pacijenta na specijalistički pregled/postupak }}
					\begin{packed_item}
	
						\item \textbf{Glavni sudionik: }pedijatar, liječnik obiteljske medicine
						\item  \textbf{Cilj:} naručiti pacijenta na specijalistički pregled/postupak
						\item  \textbf{Sudionici:} baza podataka
						\item  \textbf{Preduvjet:} liječnik je prijavljen u sustav te je pacijent prijavljen u njegovoj ordinaciji
						\item  \textbf{Opis osnovnog tijeka:}
						
						\item[] \begin{packed_enum}
	
							\item liječnik pregledava popis svojih pacijenata
							\item liječnik odabire pacijenta kojeg želi naručiti na specijalistički pregled/postupak
							\item liječnik stvara novi specijalistički pregled i zakaže vrijeme
                            \item vrijeme i lokacija se šalju pacijenti te je vidljiv pod 'Moji pregledi'

						\end{packed_enum}
					\end{packed_item}


                     \noindent \underbar{\textbf{UC14 - pregledaj preporuke za bolovanje}}
					\begin{packed_item}
	
						\item \textbf{Glavni sudionik: }liječnik obiteljske medicine
						\item  \textbf{Cilj:} pregledati preoporuku za bolovanje roditelja izdanu od strane pedijatra
						\item  \textbf{Sudionici:} baza podataka
						\item  \textbf{Preduvjet:} liječnik je prijavljen u sustav, pedijatar je izdao preporuku za bolovanje
						\item  \textbf{Opis osnovnog tijeka:}
						
						\item[] \begin{packed_enum}
	
							\item liječnik pregledava listu preporuka za bolovanje njegovim pacijentima
							\item liječnik odabire preporuku koju želi pregledati
							\item liječnik preporuku može odbiti ili prihvatiti
                            \item u slučaju da je liječnik odobrio preporuku, doznaka za bolovanje se šalje poslodavcu pacijenta

						\end{packed_enum}
					\end{packed_item}

                    
                    \noindent \underbar{\textbf{UC15 - pregledaj korisnika}}
					\begin{packed_item}
	
						\item \textbf{Glavni sudionik: }adminsitrator
						\item  \textbf{Cilj:} pregledati registrirane korisnike
						\item  \textbf{Sudionici:} baza podataka
						\item  \textbf{Preduvjet:} administrator je prijavljen
						\item  \textbf{Opis osnovnog tijeka:}
						
						\item[] \begin{packed_enum}
	
							\item administrator odabire opciju pregledavanja korisnika
							\item prikaže se lista svih ispravno registriranih korisnika s osobnim podacima

						\end{packed_enum}
					\end{packed_item}

                    \noindent \underbar{\textbf{UC15 - obriši korisnika}}
					\begin{packed_item}
	
						\item \textbf{Glavni sudionik: }adminsitrator
						\item  \textbf{Cilj:} obrisati korisnika
						\item  \textbf{Sudionici:} baza podataka
						\item  \textbf{Preduvjet:} administrator je prijavljen
						\item  \textbf{Opis osnovnog tijeka:}
						
						\item[] \begin{packed_enum}
	
							\item administrator odabire opciju uklanjanja korisnika
							\item administrator pronalazi željenog korisnika
                            \item administrator uklanja željenog korisnika i njegove podatke iz baze podataka

						\end{packed_enum}
					\end{packed_item}

                    \noindent \underbar{\textbf{UC16 - promjeni prava pristupa}}
					\begin{packed_item}
	
						\item \textbf{Glavni sudionik: }adminsitrator
						\item  \textbf{Cilj:} promijeniti razinu pristupa korisnika
						\item  \textbf{Sudionici:} baza podataka
						\item  \textbf{Preduvjet:} administrator je prijavljen
						\item  \textbf{Opis osnovnog tijeka:}
						
						\item[] \begin{packed_enum}
	
							\item administrator pronalazi željenog korisnika
							\item administrator mijenja razinu pristupa željenom korisniku
                        
						\end{packed_enum}
					\end{packed_item}
        
                    \noindent \underbar{\textbf{UC17 - administriraj registre djece i korisničke račune}}
					\begin{packed_item}
	
						\item \textbf{Glavni sudionik: }administrator
						\item  \textbf{Cilj:} upravljati registrima djece i korisničkim računima u sustavu
						\item  \textbf{Sudionici:} baza podataka
						\item  \textbf{Preduvjet:} administrator je prijavljen u sustav, postoji potreba za stvaranjem ili ažuriranjem korisničkih računa i registara djece
						\item  \textbf{Opis osnovnog tijeka:}
						
						\item[] \begin{packed_enum}
	
							\item administrator se prijavljuje u sustav
							\item administrator ima ovlasti za stvaranje, uređivanje i brisanje korisničkih računa i zapisa djece
							\item administrator stvara zapise za djecu povezane s njihovim roditeljima

						\end{packed_enum}
						
					\end{packed_item}

                    \noindent \underbar{\textbf{UC18 - poveži roditelja s djetetom}}
					\begin{packed_item}
	
						\item \textbf{Glavni sudionik: }administrator
						\item  \textbf{Cilj:} povezati profil roditelja s registrom njegove djece u sustavu
						\item  \textbf{Sudionici:} baza podataka
						\item  \textbf{Preduvjet:} roditelj je registriran, postoje registri djece
						\item  \textbf{Opis osnovnog tijeka:}
						
						\item[] \begin{packed_enum}
	
							\item administrator odabire roditelja i dijete koje želi povezati
							\item administrator izvršava povezivanje roditelja i dijete u sustavu
						
						\end{packed_enum}
						
					\end{packed_item}
                    
        
                    \noindent \underbar{\textbf{UC19 - ažuriraj podatke u aplikaciji}}
					\begin{packed_item}
	
						\item \textbf{Glavni sudionik: }administrator
						\item  \textbf{Cilj:} ažurirati sve podatke u aplikaciji, uključujući podatke o korisnicima i njihovim ovlastima
						\item  \textbf{Sudionici:} baza podataka
						\item  \textbf{Preduvjet:} administrator je prijavljen u sustav
						\item  \textbf{Opis osnovnog tijeka:}
						
						\item[] \begin{packed_enum}
	
							\item administrator odabire podatke koje želi ažurirati
							\item administrator ažurira željene podatke
						
						\end{packed_enum}
					
					\end{packed_item}

                    % \noindent \underbar{\textbf{UC20 - }}
					% \begin{packed_item}
	
					% 	\item \textbf{Glavni sudionik: }
					% 	\item  \textbf{Cilj:} 
					% 	\item  \textbf{Sudionici:} 
					% 	\item  \textbf{Preduvjet:} -
					% 	\item  \textbf{Opis osnovnog tijeka:}
						
					% 	\item[] \begin{packed_enum}
	
					% 		\item 
					% 		\item 
					% 		\item 

					% 	\end{packed_enum}
						
					% 	\item  \textbf{Opis mogućih odstupanja:}
						
					% 	\item[] \begin{packed_item}
	
					% 		\item[2.a] 
					% 		\item[] \begin{packed_enum}
								
					% 			\item 
					% 			\item 
					% 		\end{packed_enum}
							
					% 	\end{packed_item}
					% \end{packed_item}
     
				\subsubsection{Dijagrami obrazaca uporabe}
					
					%Prikazati odnos aktora i obrazaca uporabe odgovarajućim UML dijagramom. Nije nužno nacrtati sve na jednom dijagramu. Modelirati po razinama apstrakcije i skupovima srodnih funkcionalnosti.
				\eject		
				
			\subsection{Sekvencijski dijagrami}
				
				%dio 1. revizije
				
				%Nacrtati sekvencijske dijagrame koji modeliraju najvažnije dijelove sustava (max. 4 dijagrama). Uz svaki dijagram napisati detaljni opis dijagrama.
    
				\eject
	
		\section{Ostali zahtjevi}
		
			%dio 1. revizije
		 
			 %Nefunkcionalni zahtjevi i zahtjevi domene primjene dopunjuju funkcionalne zahtjeve. Oni opisuju kako se sustav treba ponašati i koja ograničenja treba poštivati (performanse, korisničko iskustvo, pouzdanost, standardi kvalitete, sigurnost...). Primjeri takvih zahtjeva u Vašem projektu mogu biti: podržani jezici korisničkog sučelja, vrijeme odziva, najveći mogući podržani broj korisnika, podržane web/mobilne platforme, razina zaštite (protokoli komunikacije, kriptiranje...)... Svaki takav zahtjev potrebno je navesti u jednoj ili dvije rečenice.
			 
			 
			 
	