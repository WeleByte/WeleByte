\chapter*{Dodatak: Prikaz aktivnosti grupe}
		\addcontentsline{toc}{chapter}{Dodatak: Prikaz aktivnosti grupe}
		
		\section*{Dnevnik sastajanja}
		
		\begin{packed_enum}
			\item  sastanak
			
			\item[] \begin{packed_item}
				\item Datum: \date[{19. listopada 2023.} %%zas ovdje ide ova sintaksa [
				\item Prisustvovali: D. Matić, L. Crvelin, V. Kumanović, M. Lešković, K. Valečić, M. Vidaković
				\item Teme sastanka:
				\begin{packed_item}
					\item  izbor tehnologije
					\item  izbor teme
                    \item  ime tima
                    \item  osnovna podjela posla
                    \item  osiguravanje pristupa na sve servisa svim članovima tima
				\end{packed_item}
            \item Dogovoreni su taskovi za prvi sprint (do četvrtrka 26.10). Oni za većinu ljudi uključuju učenje gita i tehnologija koje su odabrali (backend/frontend), a L. Crvelin treba i napraviti prve korake u organizaciji dokumentacije. Voditelj D. Matić će postaviti repozitorij na GitHubu i osigurati svima pritup. Dogovorena je i radionica na temu značajki (featurea) aplikacije sutra u 12 sati, gdje ćemo pokušati prioritizirati posao i složiti Jiru.
			\end{packed_item}

   
			\item  sastanak
			\item[] \begin{packed_item}
				\item Datum: \date[{20. listopada 2023.}
				\item Prisustvovali:  D. Matić, L. Crvelin, V. Kumanović, M. Lešković, K. Valečić, M. Vidaković
				\item Teme sastanka:
				\begin{packed_item}
					\item  analiziranje značajki aplikacije i napisati iz njih epice i storije u Jiri
					\item  stvaranje dijagrama
                    \item  daljnja podjela posla
				\end{packed_item}
            \item Detaljno su analizirane značajke aplikacije i stvoreni dijagrami iz kojih se može stvoriti definicija baze podataka što će do kraja sprinta učiniti V. Kumanović i prioritizirani popis zadataka koji će u Jiri stvoriti D. Matić. Također, identificirali smo nejasnoće u samoj definiciji zadatka koje ćemo prenijeti asistentici.
            M. Vidaković će do kraja sprinta istražiti kako se pravilno postavlja Spring projekt, a K. Valečić će isto učiniti za React u čemu će mu pomoći M. Lešković.
            Dogovorene su još dvije radionice sljedeći tjedan, u četvrtak i petak na temu postavljanja Reacta i Springa.
			\end{packed_item}

            \item  sastanak
			\item[] \begin{packed_item}
				\item Datum: \date[{26. listopada 2023.}
				\item Prisustvovali:  
				\item Teme sastanka:
				\begin{packed_item}
					\item  planiranje i praćenje napretka
					\item  postavljanje backend-a prema svim pravilima i omogućavanje nesmetanog razvoja u budućnosti.
				\end{packed_item}
            \item Zaključak
			\end{packed_item}

            \item  sastanak
			\item[] \begin{packed_item}
				\item Datum: \date[{26. listopada 2023.}
				\item Prisustvovali:  
				\item Teme sastanka:
				\begin{packed_item}
					\item  planiranje i praćenje napretka
					\item  postavljanje frontend-a prema svim pravilima i omogućavanje nesmetanog razvoja u budućnosti.
				\end{packed_item}
            \item Zaključak
			\end{packed_item}
			
			
		\end{packed_enum}
		
		\eject
		\section*{Tablica aktivnosti}
			 %Doprinose u aktivnostima treba navesti u satima po članovima grupe po aktivnosti.}

			\begin{longtblr}[
					label=none,
				]{
					vlines,hlines,
					width = \textwidth,
					colspec={X[7, l]X[1, c]X[1, c]X[1, c]X[1, c]X[1, c]X[1, c]X[1, c]}, 
					vline{1} = {1}{text=\clap{}},
					hline{1} = {1}{text=\clap{}},
					rowhead = 1,
				} 
			
				\SetCell[c=1]{c}{} & \SetCell[c=1]{c}{\rotatebox{90}{\textbf{Dorian Matić}}} & \SetCell[c=1]{c}{\rotatebox{90}{\textbf{Lucia Crvelin }}} &	\SetCell[c=1]{c}{\rotatebox{90}{\textbf{Leon Cvetkovski }}} & \SetCell[c=1]{c}{\rotatebox{90}{\textbf{Vedran Kumanović }}} &	\SetCell[c=1]{c}{\rotatebox{90}{\textbf{Mihael Lešković }}} & \SetCell[c=1]{c}{\rotatebox{90}{\textbf{Krešimir Valečić }}} &	\SetCell[c=1]{c}{\rotatebox{90}{\textbf{Mia Vidaković }}} \\  
				Upravljanje projektom 		&  &  &  &  &  &  & \\ 
				Opis projektnog zadatka 	&  &  &  &  &  &  & \\ 
				
				Funkcionalni zahtjevi       &  &  &  &  &  &  &  \\ 
				Opis pojedinih obrazaca 	&  &  &  &  &  &  &  \\ 
				Dijagram obrazaca 			&  &  &  &  &  &  &  \\ 
				Sekvencijski dijagrami 		&  &  &  &  &  &  &  \\ 
				Opis ostalih zahtjeva 		&  &  &  &  &  &  &  \\ 

				Arhitektura i dizajn sustava	 &  &  &  &  &  &  &  \\ 
				Baza podataka				&  &  &  &  &  &  &   \\ 
				Dijagram razreda 			&  &  &  &  &  &  &   \\ 
				Dijagram stanja				&  &  &  &  &  &  &  \\ 
				Dijagram aktivnosti 		&  &  &  &  &  &  &  \\ 
				Dijagram komponenti			&  &  &  &  &  &  &  \\ 
				Korištene tehnologije i alati 		&  &  &  &  &  &  &  \\ 
				Ispitivanje programskog rješenja 	&  &  &  &  &  &  &  \\ 
				Dijagram razmještaja			&  &  &  &  &  &  &  \\ 
				Upute za puštanje u pogon 		&  &  &  &  &  &  &  \\  
				Dnevnik sastajanja 			&  &  &  &  &  &  &  \\ 
				Zaključak i budući rad 		&  &  &  &  &  &  &  \\  
				Popis literature 			&  &  &  &  &  &  &  \\  
				&  &  &  &  &  &  &  \\ \hline 
				\textit{Dodatne stavke kako ste podijelili izradu aplikacije} 			&  &  &  &  &  &  &  \\ 
				\textit{npr. izrada početne stranice} 				&  &  &  &  &  &  &  \\  
				\textit{izrada baze podataka} 		 			&  &  &  &  &  &  & \\  
				\textit{spajanje s bazom podataka} 							&  &  &  &  &  &  &  \\ 
				\textit{back end} 							&  &  &  &  &  &  &  \\  
				 							&  &  &  &  &  &  &\\ 
			\end{longtblr}
					
					
		\eject
		\section*{Dijagrami pregleda promjena}
		
		%dio 2. revizije
		
		%Prenijeti dijagram pregleda promjena nad datotekama projekta. Potrebno je na kraju projekta generirane grafove s gitlaba prenijeti u ovo poglavlje dokumentacije. Dijagrami za vlastiti projekt se mogu preuzeti s gitlab.com stranice, u izborniku Repository, pritiskom na stavku Contributors.
		
	